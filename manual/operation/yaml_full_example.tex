This example shows a \cxoneflow configuration using all default options defined
as a handler when using a single SCM.  This is the same as one of the simple examples if
an example had explicitly defined omitted default connection elements.

The \texttt{scan-config} element has been added to this configuration to
demonstrate some of the controls that can be implemented over scan options.  In this
example, static Project and Scan tags are defined.  Also defined is the selection
of engines for the scan with some options defined as supported by the engine.  The
option keys are defined by the
\href{https://checkmarx.stoplight.io/docs/checkmarx-one-api-reference-guide/branches/main/f601dd9456e80-run-a-scan}{Scan REST API}
and described by
\href{https://checkmarx.com/resource/documents/en/34965-68598-global-settings.html#UUID-8e38f06b-45d4-ea7f-5ff5-50deb22e43aa_UUID-1a4211ec-dbf9-a180-cb20-59e1246ec3fb}{Scanners Settings}.


\begin{code}{Full YAML Configuration Example}{[CxOne: oauth]}{[SCM: token auth]}
secret-root-path: /run/secrets
server-base-url: https://cxoneflow.mydomain.com:8443/

bbdc:
    - service-name: BitBucket DC
      repo-match: .*
      scan-config:
          default-scan-engines:
              sca:
                  exploitablePath: "True"
              sast:
                  presetName: ASA Premium
                  incremental: "False"
                  fastScanMode: "True"
                  filter: "!**/node_modules,!**/test*"
                  languageMode: multi
              kics:
              apisec:
          default-scan-tags:
              scan-service: BitBucket DC
          default-project-tags:
              onboarded-by: CxOneFlow
      connection:
          base-url: https://scm.corp.com
          shared-secret: scm-shared-secret
          timeout-seconds: 60
          retries: 3
          proxies:
            http: http://proxy.corp.com:8080
            https: http://proxy.corp.com:8080
          api-auth:
              token: scm-token-secret
      cxone:
          tenant: mytenant
          oauth:
              client-id: my-oauth-id
              client-secret: my-oauth-secret
          iam-endpoint: US
          api-endpoint: US
          timeout-seconds: 60
          retries: 3
          proxies:
            http: http://proxy.corp.com:8080
            https: http://proxy.corp.com:8080
\end{code}

\pagebreak
\noindent\\The next example shows a configuration where \cxoneflow has endpoint handlers for both
BitBucket Data Center and Azure DevOps Enterprise.  Each SCM is configured to handle multiple distinct
projects to demonstrate the use of multiple authentication methods.  All the SCM endpoints
orchestrate scans in a single Checkmarx One tenant.

\begin{code}{Multi-SCM YAML Configuration Example}{}{}
secret-root-path: /run/secrets
server-base-url: https://cxoneflow.mydomain.com:8443/
adoe-connection: &adoe-con
    base-url: http://adoe.scm.org/
    shared-secret: scm-shared-secret
bbdc-connection: &bbdc-con
    base-url: http://bbdc.scm.org
    shared-secret: scm-shared-secret
adoe:
    - service-name: ADO-AsSSH
        repo-match: .*MySSHProject
        connection:
        <<: *adoe-con
        api-auth: 
            token: adoe-token-secret
        clone-auth: &clone-ssh
            ssh: ssh-priv-key
        cxone: &cxone
        tenant: my_tenant
        oauth:
            client-id: prod_client_id
            client-secret: prod_client_secret
        iam-endpoint: US
        api-endpoint: US
    - service-name: ADO-AsToken
        repo-match: .*MyTokenProject
        connection:
        <<: *adoe-con
        api-auth:
            token: adoe-token-secret
        cxone: *cxone
    - service-name: ADO-AsBasicAuth
        repo-match: .*MyBasicAuthProject
        connection:
        <<: *adoe-con
        api-auth:
            username: username-secret
            password: password-secret
        cxone: *cxone
bbdc:
    - service-name: BBDC-AsSSH
        repo-match: .*SP
        connection:
        <<: *bbdc-con
        api-auth: 
            token: bbdc-ssh-proj-pat
        clone-auth: *clone-ssh
        cxone: *cxone
    - service-name: BBDC-AsToken
        repo-match: .*TOK
        connection:
        <<: *bbdc-con
        api-auth:
            token: bbdc-token-proj-pat
        cxone: *cxone
\end{code}
