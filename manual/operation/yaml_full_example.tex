This example shows a \cxoneflow configuration explicitly setting default options in a service 
configuration for a single SCM.  The minimal examples leave several of these options as default.

The \texttt{scan-config} element has been added to this configuration to
demonstrate some of the controls that can be implemented over scan options.  In this
example, static Project and Scan tags are defined.  Also defined is the selection
of engines for the scan with some options defined as supported by the engine.
Documentation for the option keys can be found in the Checkmarx
\extlink{https://checkmarx.stoplight.io/docs/checkmarx-one-api-reference-guide/branches/main/f601dd9456e80-run-a-scan}{Scan REST API}
documentation and have the descriptions documented in the
\extlink{https://docs.checkmarx.com/en/34965-324311-settings-for-specific-scanners.html}{Scanners Settings}
documentation.

While there are options to apply scan configurations via \texttt{scan-config} elements, it is often the case that defining the scan configuration
in \cxoneflow will have undesirable results.  When defined in the \cxoneflow configuration, the configuration will explicitly override \cxone
tenant and project level default scan configurations.  Details about utilizing the \cxone configuration options for best results with \cxoneflow
can be found in Section \ref{sec:deployment-scan-defaults}.

\begin{code}{Full YAML Configuration Example}{[CxOne: oauth]}{[SCM: token auth]}
secret-root-path: /run/secrets
server-base-url: https://cxoneflow.mydomain.com:8443/

bbdc:
    - service-name: BitBucket DC
      repo-match: .*
      scan-config:
          default-scan-engines:
              sca:
                  exploitablePath: "True"
              sast:
                  presetName: ASA Premium
                  incremental: "False"
                  fastScanMode: "True"
                  filter: "!**/node_modules,!**/test*"
                  languageMode: multi
              kics:
              apisec:
          default-scan-tags:
              scan-service: BitBucket DC
          default-project-tags:
              onboarded-by: CxOneFlow
      connection:
          base-url: https://scm.corp.com
          shared-secret: scm-shared-secret
          timeout-seconds: 60
          retries: 3
          proxies:
            http: http://proxy.corp.com:8080
            https: http://proxy.corp.com:8080
          api-auth:
              token: scm-token-secret
      cxone:
          tenant: mytenant
          oauth:
              client-id: my-oauth-id
              client-secret: my-oauth-secret
          iam-endpoint: US
          api-endpoint: US
          timeout-seconds: 60
          retries: 3
          proxies:
            http: http://proxy.corp.com:8080
            https: http://proxy.corp.com:8080
\end{code}


\pagebreak
The next example shows a configuration where \cxoneflow has endpoint handlers for both
BitBucket Data Center and Azure DevOps Enterprise.  Each SCM is configured to handle multiple distinct
projects to demonstrate the use of multiple authentication methods.  All the SCM endpoints
orchestrate scans in a single \cxone tenant.

\begin{code}{Multi-SCM/Multi-Org YAML Configuration Example}{}{}
secret-root-path: /run/secrets
server-base-url: https://cxoneflow.mydomain.com:8443/
adoe-connection: &adoe-con
    base-url: http://adoe.scm.org/
    shared-secret: scm-shared-secret
bbdc-connection: &bbdc-con
    base-url: http://bbdc.scm.org
    shared-secret: scm-shared-secret
adoe:
    - service-name: ADO-EastCoast
        repo-match: .*East
        connection:
        <<: *adoe-con
        api-auth: 
            token: adoe-token-secret
        clone-auth: &clone-ssh
            ssh: ssh-priv-key
        cxone: &cxone
        tenant: my_tenant
        oauth:
            client-id: prod_client_id
            client-secret: prod_client_secret
        iam-endpoint: US
        api-endpoint: US
    - service-name: ADO-WestCoast
        repo-match: .*West
        connection:
        <<: *adoe-con
        api-auth:
            token: adoe-token-secret
        cxone: *cxone
bbdc:
    - service-name: BBDC-EastCoast
        repo-match: .*EAS
        connection:
        <<: *bbdc-con
        api-auth: 
            token: bbdc-token
        clone-auth: *clone-ssh
        cxone: *cxone
    - service-name: BBDC-WestCoast
        repo-match: .*WES
        connection:
        <<: *bbdc-con
        api-auth:
            token: bbdc-token
        cxone: *cxone
\end{code}
