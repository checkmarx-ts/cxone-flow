\chapter{Quickstart}

To get up and running with a basic configuration of \cxoneflow, these are the following steps:


\section{Step 1: Create a basic configuration YAML file}

The full configuration documentation can be found in Section \ref{sec:op-config}. To fully configure the \cxoneflow you'll need
the following information that will be used when composing the configuration YAML file:

\begin{itemize}
    \item A CheckmarxOne API Key or OAuth client id + client secret.
    \item Credentials appropriate for communicating with the Git server's API.
    \item Credentials used to clone Git repositories if the clone credentials must be different than the credentials used for API access.
\end{itemize}

\noindent\\The types of credentials may vary based on how the SCM server is configured or how \cxoneflow
integrates with the SCM.

\section{Step 2: Execute the \cxoneflow Container}

The published \cxoneflow container can be executed with the proper runtime configuration.  Please refer
to Section \ref{sec:runtime-config} for instructions about starting the \cxoneflow container instance with
the proper runtime configuration. Section \ref{sec:deployment} has additional deployment information that
may be useful in choosing how and where to host the \cxoneflow instance.


\section{Step 3: Configure SCM Webhooks}

Please refer to the chapter for your SCM platform in Part \ref{part:scms} for details about webhook configuration.


