\chapter{Quickstart}

To get up and running with a basic configuration of \cxoneflow, these are the following steps:


\section{Step 1: Create a basic configuration YAML file}

The full configuration documentation can be found in Section \ref{sec:op-config}. The following example shows a minimal \cxoneflow configuration that defines the following:

\begin{enumerate}
    \item Files containing secrets are located at \texttt{/run/secrets}.
    \item One BitBucket Data Center SCM connection configuration to handle all webhook payloads
    POSTed to the \texttt{/bbdc} endpoint.
    \item One catch-all route for clone-urls using the regular expression \texttt{.*}
    \item The SCM's base URL located at \texttt{https://scm.corp.com}
    \item The shared secret used to validate webhook payloads located in the file \texttt{/run/secrets/scm-shared-secret}
    \item The API and clone authorization using a PAT in a file located at \texttt{/run/secrets/scm-token-secret}
    \item The CheckmarxOne tenant name of \texttt{mytenant}
    \item The CheckmarxOne API credentials using an \texttt{oauth} client with:
    \begin{enumerate}
        \item The client identifier located in the file \texttt{/run/secrets/my-oauth-id}
        \item The client secret located in the file \texttt{/run/secrets/my-oauth-secret}
    \end{enumerate}
    \item Using the CheckmarxOne multi-tenant US region IAM endpoint.
    \item Using the CheckmarxOne multi-tenant US region API endpoint.
\end{enumerate}

\begin{code}{Minimal YAML Configuration Example \#1}{[CxOne: oauth]}{[SCM: token auth]}
secret-root-path: /run/secrets

bbdc:
    - service-name: BitBucket DC
      repo-match: .*
      connection:
        base-url: https://scm.corp.com
        shared-secret: scm-shared-secret
        api-auth:
            token: scm-token-secret
      cxone:
        tenant: mytenant
        oauth:
            client-id: my-oauth-id
            client-secret: my-oauth-secret
        iam-endpoint: US
        api-endpoint: US
\end{code}

\pagebreak
\noindent\\An alternate minimal example using different authorization options:

\begin{code}{Minimal YAML Configuration Example \#2}{[CxOne: api-key]}{[SCM: basic/ssh auth]}
secret-root-path: /run/secrets

bbdc:
    - service-name: BitBucket DC
      repo-match: .*
      connection:
      base-url: https://scm.corp.com
      shared-secret: scm-shared-secret
      api-auth:
          username: scm-username-secret
          password: scm-password-secret
      clone-auth:
          ssh: scm-ssh-key-secret
      cxone:
        tenant: mytenant
        api-key: my-cxone-api-key
        iam-endpoint: US
        api-endpoint: US
\end{code}
    
\pagebreak
\noindent\\An alternate minimal example using for an Azure DevOps Enterprise
SCM:

\begin{code}{Minimal YAML Configuration Example \#2}{[CxOne: api-key]}{[SCM: basic/ssh auth]}
secret-root-path: /run/secrets

adoe:
    - service-name: MyADO
      repo-match: .*
      connection:
      base-url: https://scm.corp.com
      shared-secret: scm-shared-secret
      api-auth:
          username: scm-username-secret
          password: scm-password-secret
      clone-auth:
          ssh: scm-ssh-key-secret
      cxone:
        tenant: mytenant
        api-key: my-cxone-api-key
        iam-endpoint: US
        api-endpoint: US
\end{code}


\noindent\\To fully configure the YAML, you'll need the following information:

\begin{itemize}
    \item A CheckmarxOne API Key or OAuth client id + client secret.
    \item A username/password or Personal Access Token used to communicate with the Git server's API.
    \item A username/password, Personal Access Token, or SSH private key used to clone Git repositories.
\end{itemize}


\section{Step 2: Execute the \cxoneflow Container}

The published \cxoneflow container can be executed with the proper runtime configuration.  Please refer
to Section \ref{sec:runtime-config} for instructions about starting the \cxoneflow container instance with
the proper runtime configuration.


\section{Step 3: Configure SCM Webhooks}

Please refer to the chapter in Part \ref{part:scms} regarding your SCM's webhook configuration.
