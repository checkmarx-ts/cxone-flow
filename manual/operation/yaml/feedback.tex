\subsubsection{YAML Element: <scm moniker>.feedback.pull-request}\label{sec:yaml-feedback-pull-request}
The configuration parameters for pull request feedback workflows.

\subsubsection{YAML Element: <scm moniker>.feedback.pull-request.enabled}\label{sec:yaml-pull-request-enabled}
Defaults to \texttt{False}.  If set to \texttt{True}, the feedback workflow for Pull Requests is executed upon completion of a scan generated by
a Pull Request. See Section \ref{sec:pull-request-workflow} for details about the Pull Request feedback workflow.


\subsubsection{YAML Element: <scm moniker>.feedback.scan-monitor}\label{sec:yaml-feedback-scan-monitor}
The parameters used when monitoring scan progress during workflow orchestration. 

Scan progress is monitored by requesting a scan state from the \cxone API at
a time interval.  The initial time interval is set to the value configured for
\texttt{poll-interval-seconds}.  If the scan is not found to have finished executing
at any given poll execution, the previous poll interval time is multiplied by
the scalar given in the \texttt{poll-backoff-multiplier} value up to a maximum
poll interval time configured by \texttt{poll-max-interval-seconds}.

If a scan does not finish executing by the time set in \texttt{scan-timeout-hours}, the
workflow is aborted.  The value of 0 configured for \texttt{scan-timeout-hours} indicates
the workflow will wait forever for the scan to finish executing.

\subsubsection{YAML Element: <scm moniker>.feedback.exclusions}\label{sec:yaml-feedback-exclusions}
Settings for excluding results from results from feedback output.  Each of the elements is a list that 
can be configured with multiple exclusion elements.
