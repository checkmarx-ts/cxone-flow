\subsubsection{YAML Element: Root}\label{sec:yaml-root}

The root elements of the YAML configuration are formatted to the left-most
position in the YAML file.  Anchor elements may be defined at the root but
must not clash with the names of any of the root elements.

\subsubsection{YAML Element: secret-root-path}\label{sec:yaml-secret-root-path}

A string that is the path to a directory that contains one or more files containing secret values.  The names to these files are 
referenced elsewhere in the YAML configuration file when used in a field that is a reference to a secret.

\subsubsection{YAML Element: server-base-url}\label{sec:yaml-server-base-url}
A string that is the base URL for the \cxoneflow endpoint.  This is used when creating feedback content that loads image elements.

\subsubsection{YAML Element: <scm moniker>}\label{sec:yaml-scm-monikers}

This is a moniker indicating the a list of service definitions for handling events from an SCM matching the name of the SCM
moniker.  Each service definition is a YAML dictionary of elements. The contents for each service definition dictionary 
have the same meaning unless otherwise specified.  At lease one SCM moniker with one configured service definition is required. 
The following SCM configuration monikers are currently supported:

\begin{itemize}
    \item \textbf{\texttt{bbdc}} for BitBucket Data Center webhook payloads targeting the \texttt{/bbdc}
    webhook payload receiver endpoint.
    \item \textbf{\texttt{adoe}} for Azure DevOps Enterprise or Cloud webhook payloads targeting the \texttt{/adoe}
    webhook payload receiver endpoint.
    \item \textbf{\texttt{gh}} for GitHub Enterprise or Cloud webhook payloads targeting the \texttt{/gh}
    webhook payload receiver endpoint.
\end{itemize}

