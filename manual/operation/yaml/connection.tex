\subsubsection{YAML Element: <scm moniker>.connection.api-auth}\label{sec:yaml-connection-api-auth}
A YAML dictionary of SCM authorization options for using the API.

The authorization methods for \texttt{api-auth} 
are used to communicate with the SCM's API and can often be used for cloning repositories.  The
main difference between \texttt{api-auth} and \texttt{clone-auth} is that API access generally
does not support SSH authorization. If there is a need to clone using SSH, configure the SSH
authorization under the \texttt{clone-auth} element.  

The elements of \texttt{api-auth} are required depending on the type of authorization that
needs to be performed.

\subsubsection{YAML Element: <scm moniker>.connection.api-url-suffix}\label{sec:yaml-connection-api-url-suffix}
An optional API URL suffix used when composing API request URLs. Most SCMs will not require this setting.

\subsubsection{YAML Element: <scm moniker>.connection.base-url}\label{sec:yaml-connection-base-url}
The base url of the SCM server's API endpoint. This should be the root URL for the API that can be used when composing all
API calls related to the source of the received webhook event.

\subsubsection{YAML Element: <scm moniker>.connection.base-display-url}\label{sec:yaml-connection-base-display-url}
An optional URL for use when composing links as part of an information display such as pull-request
feedback. Most SCMs will not require this setting.

\subsubsection{YAML Element: <scm moniker>.connection.clone-auth}\label{sec:yaml-connection-clone-auth}
Defines authorization options for performing clones when it differs from authorization for API requests.

The \texttt{clone-auth} element is optional;  if not provided, the connection information defined
in \texttt{api-auth} will be used.

\subsubsection{YAML Element: <scm moniker>.connection.shared-secret}\label{sec:yaml-connection-shared-secret}
The shared secret configured in the SCM used to sign webhook payloads. The shared secret must meet the
following minimum criteria: 

\begin{itemize}
  \item 20 characters long
  \item contains at least 3 numbers
  \item contains at least 3 upper-case letters
  \item contains at least 2 special characters
\end{itemize}
