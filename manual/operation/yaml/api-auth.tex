\subsubsection{YAML Element: token}\label{sec:yaml-api-auth-token}
The value specifies a file name found under the path defined by \texttt{secret-root-path} 
containing a Personal Access Token (PAT).  This is required for token authorization.  This can
be combined with the \texttt{username} element.


\subsubsection{YAML Element: username}\label{sec:yaml-api-auth-username}

\textbf{For Token Authorization}
The value specifies a file name found under the path defined by \texttt{secret-root-path} containing a username associated with the PAT.
This is optional and only used during cloning; if not supplied, the default username of \texttt{git} is used. Can be combined with 
the \texttt{token} element.


\textbf{For Basic Authorization}\footnote{Many SCMs no longer support basic authorization.}
The value specifies a file name found under the path defined by \texttt{secret-root-path} containing the username associated with the account used
for authorization.  This element can be supplied with the \texttt{password} element.

\subsubsection{YAML Element: password}\label{sec:yaml-api-auth-password}
The value specifies a file name found under the path defined by \texttt{secret-root-path} 
containing a password associated with the username found in the \texttt{username} element.  
This is required for basic authorization if basic authorization is supported by the SCM instance.
This can be combined with the \texttt{username} element.


\subsubsection{YAML Element: app-private-key}\label{sec:yaml-api-auth-app-private-key}
The value specifies a file name found under the path defined by \texttt{secret-root-path} containing a private key used
when obtaining application authorization. 

Application Authorization is available for use with select SCM types. Refer to Part \ref{part:scms} for details about SCMs
that support this type of authorization. When using Application Authorization, there is typically not a need to provide a separate 
method of authorization for cloning defined in the \texttt{clone-auth} element. 
