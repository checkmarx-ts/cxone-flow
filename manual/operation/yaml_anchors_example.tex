This example demonstrates defining common connection parameters that can be applied
to all connection definitions:

\begin{code}{Compacted Full YAML Configuration Example}{[CxOne: oauth]}{[SCM: token auth]}
secret-root-path: /run/secrets
server-base-url: https://cxoneflow.mydomain.com:8443/
my-connection-params: &common-connection-params
    timeout-seconds: 60
    retries: 3
    proxies:
        http: http://proxy.corp.com:8080
        https: http://proxy.corp.com:8080


bbdc:
    - service-name: BitBucket DC
      repo-match: .*
      feedback:
        pull-request:
            enabled: True
      scan-config:
          default-scan-engines:
              sca:
                  exploitablePath: True
              sast:
                  presetName: ASA Premium
                  incremental: False
                  fastScanMode: True
                  filter: "!**/node_modules,!**/test*"
                  languageMode: multi
              kics:
              apisec:
          default-scan-tags:
              scan-service: BitBucket DC
          default-project-tags:
              onboarded-by: CxOneFlow
      connection:
          base-url: https://scm.corp.com
          shared-secret: scm-shared-secret
          api-auth:
              token: scm-token-secret
          <<: *common-connection-params
      cxone:
          tenant: mytenant
          oauth:
              client-id: my-oauth-id
              client-secret: my-oauth-secret
          iam-endpoint: US
          api-endpoint: US
          <<: *common-connection-params
\end{code}


\noindent\\It is common to see a scenario where there are multiple organizations
using the same SCM instance.  A single \cxoneflow instance can be configured to accept
webhook events from all repos in each organization by using the \texttt{repo-match}
regular expression.  When a webhook payload is received, the \texttt{repo-match}
regular expression is applied to the clone URI until a match is found.

\noindent\\The example YAML below is used to demonstrate how \cxoneflow could be configured
for mulitple organizations in a single SCM. In the example, YAML anchors are utilized to 
re-use the common settings for each SCM organization.  Each organization, in this case, 
exists in the same SCM server and shares the same Checkmarx One instance.

\begin{code}{SCM Multi-Org YAML Configuration Example}{[CxOne: oauth]}{[SCM: token auth]}
secret-root-path: /run/secrets
server-base-url: https://cxoneflow.mydomain.com:8443/
my-connection-params: &common-connection-params
    timeout-seconds: 60
    retries: 3
    proxies:
    http: http://proxy.corp.com:8080
    https: http://proxy.corp.com:8080

bbdc:
    - service-name: BBDC-West
      repo-match: .*west
      scan-config: 
          default-scan-engines: &common-engine-config
              sca:
                  exploitablePath: True
              sast:
                  presetName: ASA Premium
                  incremental: False
                  fastScanMode: True
                  filter: "!**/node_modules,!**/test*"
                  languageMode: multi
              kics:
              apisec:
          default-scan-tags:
              scan-service: BBDC-West
          default-project-tags:
              onboarded-by: CxOneFlow
              region: West
      connection:
          base-url: https://scm.corp.com
          shared-secret: scm-west-org-shared-secret
          api-auth:
              token: scm-token-secret
          <<: *common-connection-params
      cxone: &cxone-config
          tenant: mytenant
          oauth:
              client-id: my-oauth-id
              client-secret: my-oauth-secret
          iam-endpoint: US
          api-endpoint: US
          <<: *common-connection-params
    - service-name: BBDC-East
      repo-match: .*east
      scan-config: 
          default-scan-engines: *common-engine-config
          default-scan-tags:
              scan-service: BBDC-East
          default-project-tags:
              onboarded-by: CxOneFlow
              region: East
      connection:
          base-url: https://scm.corp.com
          shared-secret: scm-east-org-shared-secret
          api-auth:
              token: scm-token-secret
          <<: *common-connection-params
      cxone: *cxone-config
\end{code}

