\chapter{Deployment}\label{sec:deployment}

If you've created a YAML configuration (Section \ref{sec:op-config}), configured your SCM to emit webhook events
(Part \ref{part:scms}) and know your runtime configuration requirements (Section \ref{sec:runtime-config}) then
you are ready to deploy.

\section{Obtaining the Container Image}

\cxoneflow is published as a container image.  If using Docker, the latest image can be pulled and cached locally
with the following command:

\begin{code}{Caching the \cxoneflow Container Image Locally}{}{}
docker pull ghcr.io/checkmarx-ts/cxone/cxone-flow:latest
\end{code}

\noindent\\While \cxoneflow is open source, making your own modified image may be difficult to support.  All
support for \cxoneflow is provided by Checkmarx Professional Services.

\section{\cxoneflow Execution}

The following is an example of a command where \cxoneflow starts and is ready to accept webhook events:

\begin{code}{\cxoneflow Example Execution}{}{}
docker run \
    -v $(pwd)/config.yaml:/opt/cxone/config.yaml \
    -v $(pwd)/secrets:/run/secrets -p 8000:8000 --rm -it \ 
    ghcr.io/checkmarx-ts/cxone/cxone-flow:latest
\end{code}

\noindent\\This executes the container with your defined \texttt{config.yml}, mapping a local directory
containing secrets files to \texttt{/run/secrets}, and exposing port 8000 locally for receipt of 
webhook payloads.


\section{Deployment Considerations}

The \cxoneflow container is stateless, listens on port 8000 for unencrypted webhook event delivery, 
and listens on port 8443 for encrypted webhook event delivery.  It is possible to map port 80 and 443
externally to the internal container ports.

The number of CPU cores on the host where the container is executing may be over-allocated to other
containers or processes.  Having the container use a high number of worker processes for an over-allocated
host will make the performance worse.  It is suggested that the \cxoneflow instances are scaled
to run on different physical hosts to ensure availability.  This will mean you'll need to place the
\cxoneflow host endpoints behind a load balancer.  The \cxoneflow endpoint \texttt{/ping} is available
for monitoring each running instance to ensure it is alive.



