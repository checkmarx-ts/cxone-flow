\chapter{Deployment}\label{sec:deployment}

If you've created a YAML configuration (Section \ref{sec:op-config}), configured your SCM to emit webhook events
(Part \ref{part:scms}) and know your runtime configuration requirements (Section \ref{sec:runtime-config}) then
you are ready to deploy.

\section{Obtaining the Container Image}

\cxoneflow is published as a container image.  If using Docker, the
latest\footnote{Previous versions are also available in the
checkmarx-ts \extlink{https://github.com/orgs/checkmarx-ts/packages}{package repository}.} image can be pulled and cached locally
with the following command:

\begin{code}{Caching the \cxoneflow Container Image Locally}{}{}
docker pull ghcr.io/checkmarx-ts/cxone/cxone-flow:latest
\end{code}

\noindent\\While \cxoneflow is open source, making your own modified image may be difficult to support.  All
support for \cxoneflow is provided by Checkmarx Professional Services.

\section{\cxoneflowtext\space Execution}

The following is an example of a command where \cxoneflow starts and is ready to accept webhook events:

\begin{code}{\cxoneflow Example Execution}{}{}
docker run \
    -v $(pwd)/config.yaml:/opt/cxone/config.yaml \
    -v $(pwd)/secrets:/run/secrets -p 8000:8000 --rm -it \ 
    ghcr.io/checkmarx-ts/cxone/cxone-flow:latest
\end{code}

\noindent\\This executes the container with your defined \texttt{config.yml}, mapping a local directory
containing secrets files to \texttt{/run/secrets}, and exposing port 8000 locally for receipt of 
unencrypted webhook payloads.




\section{Deployment Considerations}

\subsection{Hosting}

The \cxoneflow container is stateless, listens on port 8000 for unencrypted webhook event delivery, 
and listens on port 8443 for encrypted webhook event delivery.  It is possible to map port 80 and 443
externally to the internal container ports.  A default self-signed SSL certificate will be used
for encrypted traffic via port 8443 unless a custom SSL certificate configuration is provided.  

The number of CPU cores on the host where the container is executing may be over-allocated to other
containers or processes.  Having the container use a high number of worker processes for an over-allocated
host will degrade performance.  The operation of \cxoneflow is not computationally intense
but it does perform rapid network and disk I/O when cloning source code and communicating with
remote system APIs.

It is suggested that the \cxoneflow instances are scaled to run on different physical hosts to
ensure availability.  This will mean you'll need to place the \cxoneflow host endpoints behind 
a load balancer.  The \cxoneflow endpoint \texttt{/ping} is available for monitoring each
running instance to ensure it is alive.

\subsection{Scan Configuration Defaults}\label{sec:deployment-scan-defaults}

Scan configuration defaults can be provided at the tenant-scope and project-scope
in \cxone.  Default parameters for each scan can be set in the \cxoneflow configuration
that will be applied to scans invoked as part of the \cxoneflow scan orchestration logic.
It is important to understand how the \cxone default
scan configurations work given the \cxoneflow configuration can override all other
settings based on how the tenant and project defaults are configured.

The configurations in \cxone are defined in the tenant or project scope with a
flag that allows the value to be overridden at a lower scope of configuration.  A
SAST scan default preset, for example, is generally configured at the tenant scope
as "ASA Premium" with the ability to override the preset at the project scope.  The
result is that initial scans that do not define a preset at the project scope or
at the time the scan is initiated use the tenant-scope-defined preset "ASA Premium".

Configuring the SAST preset in a project's settings to something other than
the SAST preset configured at the tenant scope will change the scan preset for that project.
If the tenant scope configuration was such that the SAST preset could not be overridden, the
option for configuring the preset would not appear in the project settings.

Configuring the SAST preset as a per-scan default in \cxoneflow will override both
the tenant and project scope settings if both are set to allow override.  The reason for this
is that the scan configuration provided to the API when a scan is initiated is considered to override
all other configuration scopes.  If a project scope configuration was such that the SAST preset could
not be overridden, then the \cxoneflow provided default preset set at the time of scan will be ignored.

The configurations at the tenant, project, and scan scopes work based on overrides.
This causes the value of a configuration to be set by the lowest level in which it is
defined. This makes it possible to utilize the \cxoneflow per-scan defaults in a few
different ways.

If there are no per-scan defaults configured in \cxoneflowns, all per-scan defaults
can be configured at the tenant or project scope.  This gives the flexibility to override 
the tenant scoped settings at the project scope; this is usually required so that
the project can configure scan settings that are specific to the project.

If per-scan defaults are configured in \cxoneflowns, project scope settings can
still be set but will be overridden by the \cxoneflow per-scan settings by default.  It will require
configuring the project settings and disabling override to prevent \cxoneflow per-scan
configurations from replacing the project scope configurations.  This may be confusing
to users who are not fully aware of how the scan configuration overrides work.

\subsubsection{\cxoneflow Inheritance of \texttt{filter} Settings}

The configuration scopes are implemented in \cxone as overrides for all
configuration elements defined at a higher scope.  The previous example of the
preset demonstrates the effectiveness of an override; a SAST preset defined
as a tenant-scope configuration is used for projects that don't explicitly define
a SAST preset.  The project configuration can then be modified to override the tenant-scope
SAST preset setting if required.

The override is ideal for most configuration elements, but not ideal for elements
such as \texttt{filter} that define the file/folder filters in each engine.  It is often
the case that a tenant-scope setting will define the most general filter settings but
each project will need additional project-specific filters.  The project-scope
configuration would need to copy the tenant-scope settings each time the tenant-scope
settings are modified.  Ideally, \texttt{filter} settings would extend the parent
setting rather than override it. \cxoneflow interprets \texttt{filter} settings for each
scan engine so that the resulting filter is composed of all values inherited from 
the all the scopes rather than overridden.

As an example, consider these filters defined in each scope:

\begin{itemize}
    \item Tenant Scope: "!**/test,!**/tests"
    \item Project Scope "!**/*.sql"
    \item \cxoneflow Per-Scan configuration: "!**/node\_modules"
\end{itemize}

After \cxoneflow applies the inheritance algorithm, the resulting file/folder filter setting at the scan scope will be:
\\\\\texttt{"!**/test,!**/tests,!**/*.sql,!**/node\_modules"}


From a maintenance perspective, configuring tenant-scope defaults
with override enabled (such as SAST preset) can be used to avoid configuring
per-scan defaults in \cxoneflow.  It is often desirable to also set engine-specific
file/folder exclusions at the tenant scope to avoid scanning common exclusions.  
This will likely make maintaining global settings easier in most cases given
it is common to set scan defaults at the project scope.
