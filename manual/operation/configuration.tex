\chapter{Configuration}


\section{Runtime Configuration}\label{sec:runtime-config}

\subsection{SSL}

\subsubsection{Trusting Self-Signed Certificates}\label{sec:self-signed-certs}

While the CheckmarxOne system uses TLS certificates signed by a public CA, it is possible that
corporate proxies use certificates signed by a private CA. If so, it is possible to
import custom CA certificates when using \cxoneflow.

\noindent\\The custom certificates must meet the following criteria:

\begin{itemize}
    \item Must be in the PEM format.
    \item Must be in a file ending with the extension .crt.
    \item Only one certificate is in the file.
    \item Must be mapped to the container path /usr/local/share/ca-certificates.
\end{itemize}


\noindent\\As an example, if using Docker, it is possible to map a local file to a file in the container with this mapping option added to the container execution command line:

\begin{code}{Custom CA Mapping Option}{[Docker]}{}
-v $(pwd)/custom-ca.pem:/usr/local/share/ca-certificates/custom-ca.crt
\end{code}

\subsubsection{Configuring SSL for the \cxoneflow Endpoint}

To make the \cxoneflow endpoint use SSL for communication, obtain an SSL certificate public/private key pair
and then set the following environment variables in the runtime environment:

\begin{table}[h]
    \caption{SSL Environment Variables}        
    \begin{tabularx}{\textwidth}{ll}
        \toprule
        \textbf{Variable} & \textbf{Description}\\
        \midrule
        \texttt{SSL\_CERT\_PATH} & \makecell[l]{The path to the server's SSL certificate in PEM format.}\\
        \midrule
        \texttt{SSL\_CERT\_KEY\_PATH} & \makecell[l]{The path to the certificate's unencrypted private key.}\\
        \bottomrule
    \end{tabularx}
\end{table}

\noindent\\If your SSL certificate is self-signed or signed with a non-public CA, you'll want
to import the custom CA as described in Section \ref{sec:self-signed-certs}.


\subsection{Rutime Control Environment Variables}

\begin{table}[h]
    \caption{Runtime Control Environment Variables}        
    \begin{tabularx}{\textwidth}{lccl}
        \toprule
        \textbf{Variable} & \textbf{Required} & \textbf{Default} & \textbf{Description}\\
        \midrule
        \texttt{CXONEFLOW\_WORKERS} & No & \texttt{(\# of CPUs / 2)} & \makecell[l]{The number of worker processes\\used for parallel execution. The\\maximum value will be\\set at \texttt{(\# of CPUs - 1)}}\\
        \midrule
        \texttt{LOG\_LEVEL} & No & \texttt{INFO} & \makecell[l]{The logging verbosity level.  Set to\\\texttt{DEBUG} for increased logging\\verbosity.}\\
        \midrule
        \texttt{CONFIG\_YAML\_PATH} & No & \texttt{/opt/cxone/config.yaml} & \makecell[l]{The path to the configuration\\YAML file.}\\
        \midrule
        \texttt{CXONEFLOW\_HOSTNAME} & No & \texttt{localhost} & \makecell[l]{The virtual hostname of the\\\cxoneflow endpoint.}\\
        \bottomrule
    \end{tabularx}
\end{table}


\newpage

\section{Operational Configuration}\label{sec:op-config}

The operational configuration is done using a YAML file mapped at \texttt{/opt/cxone/config.yaml}
by default.  

\subsection{YAML Configuration Examples}

\subsubsection{Basic YAML Configurations}

The following example shows a minimal \cxoneflow configuration that defines the following:

\begin{enumerate}
    \item Files containing secrets are located at \texttt{/run/secrets}.
    \item One BitBucket Data Center SCM connection configuration to handle all webhook payloads
    POSTed to the \texttt{/bbdc} endpoint.
    \item One catch-all route for clone-urls using the regular expression \texttt{.*}
    \item The SCM's base URL located at \texttt{https://scm.corp.com}
    \item The shared secret used to validate webhook payloads located in the file \texttt{/run/secrets/scm-shared-secret}
    \item The API and clone authorization using a PAT in a file located at \texttt{/run/secrets/scm-token-secret}
    \item The CheckmarxOne tenant name of \texttt{mytenant}
    \item The CheckmarxOne API credentials using an \texttt{oauth} client with:
    \begin{enumerate}
        \item The client identifier located in the file \texttt{/run/secrets/my-oauth-id}
        \item The client secret located in the file \texttt{/run/secrets/my-oauth-secret}
    \end{enumerate}
    \item Using the CheckmarxOne multi-tenant US region IAM endpoint.
    \item Using the CheckmarxOne multi-tenant US region API endpoint.
\end{enumerate}

\begin{code}{Minimal YAML Configuration Example \#1}{[CxOne: oauth]}{[SCM: token auth]}
secret-root-path: /run/secrets

bbdc:
    - service-name: BitBucket DC
      repo-match: .*
      connection:
        base-url: https://scm.corp.com
        shared-secret: scm-shared-secret
        api-auth:
            token: scm-token-secret
      cxone:
        tenant: mytenant
        oauth:
            client-id: my-oauth-id
            client-secret: my-oauth-secret
        iam-endpoint: US
        api-endpoint: US
\end{code}

\pagebreak
\noindent\\An alternate minimal example using different authorization options:

\begin{code}{Minimal YAML Configuration Example \#2}{[CxOne: api-key]}{[SCM: basic/ssh auth]}
secret-root-path: /run/secrets

bbdc:
    - service-name: BitBucket DC
      repo-match: .*
      connection:
      base-url: https://scm.corp.com
      shared-secret: scm-shared-secret
      api-auth:
          username: scm-username-secret
          password: scm-password-secret
      clone-auth:
          ssh: scm-ssh-key-secret
      cxone:
        tenant: mytenant
        api-key: my-cxone-api-key
        iam-endpoint: US
        api-endpoint: US
\end{code}
    
\pagebreak
\noindent\\An alternate minimal example using for an Azure DevOps Enterprise
SCM:

\begin{code}{Minimal YAML Configuration Example \#2}{[CxOne: api-key]}{[SCM: basic/ssh auth]}
secret-root-path: /run/secrets

adoe:
    - service-name: MyADO
      repo-match: .*
      connection:
      base-url: https://scm.corp.com
      shared-secret: scm-shared-secret
      api-auth:
          username: scm-username-secret
          password: scm-password-secret
      clone-auth:
          ssh: scm-ssh-key-secret
      cxone:
        tenant: mytenant
        api-key: my-cxone-api-key
        iam-endpoint: US
        api-endpoint: US
\end{code}


\newpage
\noindent\\This example shows a \cxoneflow configuration explicitly setting default options in a service 
configuration for a single SCM.  The minimal examples leave several of these options as default.

The \texttt{scan-config} element has been added to this configuration to
demonstrate some of the controls that can be implemented over scan options.  In this
example, static Project and Scan tags are defined.  Also defined is the selection
of engines for the scan with some options defined as supported by the engine.
Documentation for the option keys can be found in the Checkmarx
\extlink{https://checkmarx.stoplight.io/docs/checkmarx-one-api-reference-guide/branches/main/f601dd9456e80-run-a-scan}{Scan REST API}
documentation and have the descriptions documented in the
\extlink{https://docs.checkmarx.com/en/34965-324311-settings-for-specific-scanners.html}{Scanners Settings}
documentation.

While there are options to apply scan configurations via \texttt{scan-config} elements, it is often the case that defining the scan configuration
in \cxoneflow will have undesirable results.  When defined in the \cxoneflow configuration, the configuration will explicitly override \cxone
tenant and project level default scan configurations.  Details about utilizing the \cxone configuration options for best results with \cxoneflow
can be found in Section \ref{sec:deployment-scan-defaults}.

\begin{code}{Full YAML Configuration Example}{[CxOne: oauth]}{[SCM: token auth]}
secret-root-path: /run/secrets
server-base-url: https://cxoneflow.mydomain.com:8443/

bbdc:
    - service-name: BitBucket DC
      repo-match: .*
      scan-config:
          default-scan-engines:
              sca:
                  exploitablePath: "True"
              sast:
                  presetName: ASA Premium
                  incremental: "False"
                  fastScanMode: "True"
                  filter: "!**/node_modules,!**/test*"
                  languageMode: multi
              kics:
              apisec:
          default-scan-tags:
              scan-service: BitBucket DC
          default-project-tags:
              onboarded-by: CxOneFlow
      connection:
          base-url: https://scm.corp.com
          shared-secret: scm-shared-secret
          timeout-seconds: 60
          retries: 3
          proxies:
            http: http://proxy.corp.com:8080
            https: http://proxy.corp.com:8080
          api-auth:
              token: scm-token-secret
      cxone:
          tenant: mytenant
          oauth:
              client-id: my-oauth-id
              client-secret: my-oauth-secret
          iam-endpoint: US
          api-endpoint: US
          timeout-seconds: 60
          retries: 3
          proxies:
            http: http://proxy.corp.com:8080
            https: http://proxy.corp.com:8080
\end{code}


\pagebreak
The next example shows a configuration where \cxoneflow has endpoint handlers for both
BitBucket Data Center and Azure DevOps Enterprise.  Each SCM is configured to handle multiple distinct
projects to demonstrate the use of multiple authentication methods.  All the SCM endpoints
orchestrate scans in a single \cxone tenant.

\begin{code}{Multi-SCM/Multi-Org YAML Configuration Example}{}{}
secret-root-path: /run/secrets
server-base-url: https://cxoneflow.mydomain.com:8443/
adoe-connection: &adoe-con
    base-url: http://adoe.scm.org/
    shared-secret: scm-shared-secret
bbdc-connection: &bbdc-con
    base-url: http://bbdc.scm.org
    shared-secret: scm-shared-secret
adoe:
    - service-name: ADO-EastCoast
        repo-match: .*East
        connection:
        <<: *adoe-con
        api-auth: 
            token: adoe-token-secret
        clone-auth: &clone-ssh
            ssh: ssh-priv-key
        cxone: &cxone
        tenant: my_tenant
        oauth:
            client-id: prod_client_id
            client-secret: prod_client_secret
        iam-endpoint: US
        api-endpoint: US
    - service-name: ADO-WestCoast
        repo-match: .*West
        connection:
        <<: *adoe-con
        api-auth:
            token: adoe-token-secret
        cxone: *cxone
bbdc:
    - service-name: BBDC-EastCoast
        repo-match: .*EAS
        connection:
        <<: *bbdc-con
        api-auth: 
            token: bbdc-token
        clone-auth: *clone-ssh
        cxone: *cxone
    - service-name: BBDC-WestCoast
        repo-match: .*WES
        connection:
        <<: *bbdc-con
        api-auth:
            token: bbdc-token
        cxone: *cxone
\end{code}



\subsubsection{Complex YAML Configurations using YAML Anchors}

For complex configurations, it is possible to use 
\href{https://docs.docker.com/compose/compose-file/10-fragments/}{YAML Anchors}
to avoid repeating some section definitions.  When using YAML anchors, it may be useful
to use a \href{https://onlineyamltools.com/convert-yaml-to-json}{YAML-to-JSON} conversion tool that shows the JSON generated from the YAML
definition

\noindent\\This example demonstrates defining common connection parameters that can be applied
to all connection definitions:


\begin{code}{Compacted Full YAML Configuration Example}{[CxOne: oauth]}{[SCM: token auth]}
secret-root-path: /run/secrets

my-connection-params: &common-connection-params
    timeout-seconds: 60
    retries: 3
    ssl-verify: True
    proxies:
    http: http://proxy.corp.com:8080
    https: http://proxy.corp.com:8080


bbdc:
    - service-name: BitBucket DC
      repo-match: .*
      scan-config:
          default-scan-engines:
              sca:
                  exploitablePath: "True"
              sast:
                  presetName: ASA Premium
                  incremental: "False"
                  fastScanMode: "True"
                  filter: "!**/node_modules,!**/test*"
                  languageMode: multi
              kics:
              apisec:
          default-scan-tags:
              scan-service: BitBucket DC
          default-project-tags:
              onboarded-by: CxOneFlow
      connection:
          base-url: https://scm.corp.com
          shared-secret: scm-shared-secret
          api-auth:
              token: scm-token-secret
          <<: *common-connection-params
      cxone:
          tenant: mytenant
          oauth:
              client-id: my-oauth-id
              client-secret: my-oauth-secret
          iam-endpoint: US
          api-endpoint: US
          <<: *common-connection-params
\end{code}


\noindent\\It is common to see a scenario where there are multiple organizations
using the same SCM instance.  A single \cxoneflow instance can be configured to accept
webhook events from all repos in each organization by using the \texttt{repo-match}
regular expression.  When a webhook payload is received, the \texttt{repo-match}
regular expression is applied to the clone URI until a match is found.

\noindent\\The example YAML below is used to demonstrate how \cxoneflow could be configured
for mulitple organizations in a single SCM. In the example, YAML anchors are utilized to 
re-use the common settings for each SCM organization.  Each organization, in this case, 
exists in the same SCM server and shares the same Checkmarx One instance.

\begin{code}{SCM Multi-Org YAML Configuration Example}{[CxOne: oauth]}{[SCM: token auth]}
secret-root-path: /run/secrets

my-connection-params: &common-connection-params
    timeout-seconds: 60
    retries: 3
    ssl-verify: True
    proxies:
    http: http://proxy.corp.com:8080
    https: http://proxy.corp.com:8080

bbdc:
    - service-name: BBDC-West
      repo-match: .*west
      scan-config: 
          default-scan-engines: &common-engine-config
              sca:
                  exploitablePath: "True"
              sast:
                  presetName: ASA Premium
                  incremental: "False"
                  fastScanMode: "True"
                  filter: "!**/node_modules,!**/test*"
                  languageMode: multi
              kics:
              apisec:
          default-scan-tags:
              scan-service: BBDC-West
          default-project-tags:
              onboarded-by: CxOneFlow
              region: West
      connection:
          base-url: https://scm.corp.com
          shared-secret: scm-west-org-shared-secret
          api-auth:
              token: scm-token-secret
          <<: *common-connection-params
      cxone: &cxone-config
          tenant: mytenant
          oauth:
              client-id: my-oauth-id
              client-secret: my-oauth-secret
          iam-endpoint: US
          api-endpoint: US
          <<: *common-connection-params
    - service-name: BBDC-East
      repo-match: .*east
      scan-config: 
          default-scan-engines: *common-engine-config
          default-scan-tags:
              scan-service: BBDC-East
          default-project-tags:
              onboarded-by: CxOneFlow
              region: East
      connection:
          base-url: https://scm.corp.com
          shared-secret: scm-east-org-shared-secret
          api-auth:
              token: scm-token-secret
          <<: *common-connection-params
      cxone: *cxone-config
\end{code}



\subsection{YAML Configuration Elements}

\subsubsection{YAML Root}\label{sec:yaml-root}

The root of the YAML configuration will contain the \texttt{secret-root-path} element
and one or more unique SCM configuration monikers.  The following SCM configuration monikers
are currently supported:

\begin{itemize}
    \item \texttt{bbdc} for BitBucket Data Center webhook payloads targeting the \texttt{/bbdc}
    webhook payload receiver endpoint.
    \item \texttt{adoe} for Azure DevOps Enterprise webhook payloads targeting the \texttt{/adoe}
    webhook payload receiver endpoint.
\end{itemize}


\noindent\\The value for \texttt{secret-root-path} is the path to a directory that contains one
or more files containing secret values.  The names to these files are referenced elsewhere
in the YAML configuration file as described in
\hyperref[sec:scm-block-element]{YAML SCM Configuration Element}.


\subsubsection{YAML SCM Configuration Element}\label{sec:scm-block-element}

The SCM configuration element is the same for all SCM monikers. The element is a list with
one or more entries corresponding to a clone URL regular expression match.  The entry
that first matches the clone URL received in the webhook payload is used to configure
the workflow execution parameters.  Table \ref{tab:scm-section-keys} explains the SCM
configuration keys for each SCM configuration list entry.

\begin{table}[h]
    \caption{SCM Configuration YAML Element}  
    \label{tab:scm-section-keys}      
    \begin{tabularx}{\textwidth}{lcl}
        \toprule
        \textbf{Key} & \textbf{Required} & \textbf{Description}\\
        \midrule
        \texttt{service-name} & Yes & \makecell[l]{A moniker for the route match that is used for logging purposes.}\\
        \midrule
        \texttt{repo-match} & Yes & \makecell[l]{A regex applied to the source repository.  If the webhook payload has\\a clone URL that matches the regex, this configuration is used to\\orchestrate the scanning.}\\
        \midrule
        \texttt{scan-config} & No & \makecell[l]{Elements that define the default scan configuration.  This element\\is described in the section\\"\hyperref[sec:scan-config-element]{YAML Configuration Element: \texttt{scan-config}}"}\\
        \midrule
        \texttt{connection} & Yes & \makecell[l]{SCM connection parameters. This element\\is described in the section\\"\hyperref[sec:connection-element]{YAML Configuration Element: \texttt{connection}}"}\\
        \midrule
        \texttt{cxone} & Yes & \makecell[l]{The connection configuration for the CheckmarxOne API. This\\element is described in the section\\"\hyperref[sec:cxone-element]{YAML Configuration Element: \texttt{cxone}}"}\\
        \midrule
        \texttt{rabbit} & No & \makecell[l]{The connection configuration for an external RabbitMQ\\instance. This element is described in the section\\"\hyperref[sec:rabbit-element]{YAML Configuration Element: \texttt{rabbit}}"}\\
        \bottomrule
    \end{tabularx}
\end{table}


\paragraph{YAML Configuration Element: \texttt{scan-config} }\label{sec:scan-config-element}

\noindent\\\\The \texttt{scan-config} element, described in Table \ref{tab:scan-config-section-keys}, allows for default configurations to be applied to each scan.

\begin{table}[h]
    \caption{\texttt{scan-config} YAML Element}  
    \label{tab:scan-config-section-keys}      
    \begin{tabularx}{\textwidth}{lcl}
        \toprule
        \textbf{Key} & \textbf{Required} & \textbf{Description}\\
        \midrule
        \texttt{default-scan-engines} & No & \makecell[l]{A element that follows the format\\\texttt{<engine-name>:<engine config option dictionary>}\\corresponding to the configuration element of the\\\href{https://checkmarx.stoplight.io/docs/checkmarx-one-api-reference-guide/branches/main/f601dd9456e80-run-a-scan}{Checkmarx One scan API}.}\\
        \midrule
        \texttt{default-scan-tags} & No &  \makecell[l]{A dictionary of static key:value pairs that are assigned to\\each scan.}\\
        \midrule
        \texttt{default-project-tags} & No & \makecell[l]{A dictionary of static key:value pairs that are assigned\\to each project upon project creation.}\\
        \bottomrule
    \end{tabularx}
\end{table}


\paragraph{YAML Configuration Element: \texttt{cxone} }\label{sec:cxone-element}

\noindent\\\\The \texttt{cxone} element, described in Table \ref{tab:cxone-section-keys}, 
describes the CheckmarxOne API connection parameters.


\begin{table}[h]
    \caption{\texttt{cxone} YAML Element}  
    \label{tab:cxone-section-keys}      
    \begin{tabularx}{\textwidth}{lccl}
        \toprule
        \textbf{Key} & \textbf{Required} & \textbf{Default} & \textbf{Description}\\
        \midrule
        \texttt{tenant} & Yes & N/A & \makecell[l]{The name of the CheckmarxOne tenant.}\\
        \midrule
        \texttt{iam-endpoint} & Yes & N/A & \makecell[l]{This can be a fully qualified domain name of a server\\or a multi-tenant IAM endpoint moniker as described\\in Appendix \ref{sec:endpoint-monikers}.}\\
        \midrule
        \texttt{api-endpoint} & Yes & N/A & \makecell[l]{This can be a fully qualified domain name of a server\\or a multi-tenant API endpoint moniker as described\\in Appendix \ref{sec:endpoint-monikers}.}\\
        \midrule
        \texttt{timeout-seconds} & No & 60s & \makecell[l]{The number of seconds before a request for API\\results times out.}\\
        \midrule
        \texttt{retries} & No & 3 & \makecell[l]{The number of retries when the request fails.}\\
        \midrule
        \texttt{ssl-verify} & No & True & \makecell[l]{If False, server SSL certificates are not validated.}\\
        \midrule
        \texttt{proxies} & No & N/A & \makecell[l]{A dictionary of \texttt{<scheme>:<url>} pairs to use a proxy\\server for requests. See: \href{https://requests.readthedocs.io/en/latest/user/advanced/\#proxies}{Python "requests" proxies}.}\\
        \midrule
        \texttt{api-key} & No & N/A & \makecell[l]{If not defined, the \texttt{oauth} element must be defined.\\The value specifies a file name found under the path\\defined by \texttt{secret-root-path}.}\\
        \midrule
        \texttt{oauth} & No & N/A & \makecell[l]{If not defined, the \texttt{api-key} element must be defined.\\This contains two required elements \texttt{client-id}\\and \texttt{client-secret} where each value corresponds to\\a file name found under the path defined by\\\texttt{secret-root-path}. }\\
        \bottomrule
    \end{tabularx}
\end{table}

\paragraph{YAML Configuration Element: \texttt{rabbit} }\label{sec:rabbit-element}

\noindent\\\\The \texttt{rabbit} element, described in Table \ref{tab:rabbit-section-keys}, 
describes the RabbitMQ connection parameters.  This optional element is required only
when \cxoneflow uses an external RabbitMQ element for workflow persistence and
scaling.

The \cxoneflow container runs an instance of RabbitMQ that is used to orchestrate background
processing.  The internal instance is not publicly accessible.  Appendix \ref{sec:rabbitmq}
details how an external RabbitMQ instance can be used to persist \cxoneflow workflows.  

\begin{table}[h]
    \caption{\texttt{rabbit} YAML Element}  
    \label{tab:rabbit-section-keys}      
    \begin{tabularx}{\textwidth}{lccl}
        \toprule
        \textbf{Key} & \textbf{Required} & \textbf{Default} & \textbf{Description}\\
        \midrule
        \texttt{amqp-url} & Yes & N/A & \makecell[l]{The AMQP/AMQPS URL for the RabbitMQ instance.}\\
        \midrule
        \texttt{amqp-user} & No & N/A & \makecell[l]{If the username is not included in the AMQP URL,
        the\\provided value corresponds to a file name found under\\the path defined by
        \texttt{secret-root-path}. }\\
        \midrule
        \texttt{amqp-password} & No & N/A & \makecell[l]{If the password is not included in the AMQP URL,
        the\\provided value corresponds to a file name found under\\the path defined by
        \texttt{secret-root-path}.}\\
        \midrule
        \texttt{ssl-verify} & No & True & \makecell[l]{If \texttt{False} and connecting to RabbitMQ
        with an\\AMQPS URL, certificate validation is not performed.}\\
        \bottomrule
    \end{tabularx}
\end{table}


\pagebreak
\paragraph{YAML Configuration Element: \texttt{connection} }\label{sec:connection-element}

\noindent\\\\The \texttt{connection} element, described in Table \ref{tab:connection-section-keys}, 
describes the SCM connection parameters used for API access and cloning.


\begin{table}[h]
    \caption{\texttt{connection} YAML Element}  
    \label{tab:connection-section-keys}      
    \begin{tabularx}{\textwidth}{lccl}
        \toprule
        \textbf{Key} & \textbf{Required} & \textbf{Default} & \textbf{Description}\\
        \midrule
        \texttt{base-url} & Yes & N/A & \makecell[l]{The base url of the SCM server.}\\
        \midrule
        \texttt{shared-secret} & Yes & N/A & \makecell[l]{The shared secret configured in the SCM used to sign\\webhook payloads. The shared secret must meet the\\following minimum criteria: 20 characters long,\\contains at least 3 numbers, contains at least\\3 upper-case letters, and contains at least 2 special\\characters.}\\
        \midrule
        \texttt{timeout-seconds} & No & 60s & \makecell[l]{The number of seconds before a request for API\\results times out.}\\
        \midrule
        \texttt{retries} & No & 3 & \makecell[l]{The number of retries when the request fails.}\\
        \midrule
        \texttt{ssl-verify} & No & True & \makecell[l]{If False, server SSL certificates are not validated.}\\
        \midrule
        \texttt{proxies} & No & N/A & \makecell[l]{A dictionary of \texttt{<scheme>:<url>} pairs to use a proxy\\server for requests. See: \href{https://requests.readthedocs.io/en/latest/user/advanced/\#proxies}{Python "requests" proxies}.}\\
        \midrule
        \texttt{api-auth} & Yes & N/A & \makecell[l]{A dictionary of SCM authorization options\\for using the API.\\See: \hyperref[sec:api-auth-element]{YAML Configuration Element: \texttt{api-auth}}}\\
        \midrule
        \texttt{clone-auth} & No & \makecell[l]{\texttt{api-auth}} & \makecell[l]{Authorization options for performing clones when it\\differs from authorization for API requests.\\See: \hyperref[sec:clone-auth-element]{YAML Configuration Element: \texttt{clone-auth}}}\\
        \bottomrule
    \end{tabularx}
\end{table}

\pagebreak
\paragraph{YAML Configuration Element: \texttt{clone-auth} }\label{sec:clone-auth-element}

\noindent\\\\The \texttt{clone-auth} element is optional;  if not provided, the connection information defined
in \texttt{api-auth} will be used.  This element can contain the following key:value pair combinations:

\noindent\\\textbf{Token Authorization Elements}

\noindent\\To clone with a token, the following elements can appear under the \texttt{clone-auth}
element exclusive of other elements:

\begin{itemize}
    \item \texttt{token} - The value specifies a file name found under the path defined
    by \texttt{secret-root-path} containing a Personal Access Token (PAT).  This is required for
    token authorization.
    \item \texttt{username} - The value specifies a file name found under the path defined
    by \texttt{secret-root-path} containing a username associated with the PAT.  This is 
    optional; if not supplied, the default username of \texttt{git} is used.
\end{itemize}

\noindent\\\textbf{Basic Authorization Elements}

\noindent\\To clone with basic authorization, the following required elements can appear under the
\texttt{clone-auth} element exclusive of other elements:

\begin{itemize}
    \item \texttt{username} - The value specifies a file name found under the path defined
    by \texttt{secret-root-path} containing the username associated with the account used
    for authorization. 
    \item \texttt{password} - The value specifies a file name found under the path defined
    by \texttt{secret-root-path} containing the password associated with the account used
    for authorization. 
\end{itemize}

\noindent\\\textbf{SSH Authorization Elements}

\noindent\\To clone with SSH, the following required element can appear under the
\texttt{clone-auth} element exclusive of other elements:

\begin{itemize}
    \item \texttt{ssh} - The value specifies a file name found under the path defined
    by \texttt{secret-root-path} containing an unencrypted private SSH key.
    \item \texttt{ssh-port} - This optional value specifies the port used for SSH cloning
    if the SCM is not using port 22 and does not automatically include it in the clone
    URL.
\end{itemize}

\paragraph{YAML Configuration Element: \texttt{api-auth} }\label{sec:api-auth-element}

\noindent\\\\The \texttt{api-auth} element is required.  The authorization methods for \texttt{api-auth} 
are used to communicate with the SCM's API and can often be used for cloning repositories.  The
main difference between \texttt{api-auth} and \texttt{clone-auth} is that API access generally
does not support SSH authorization. If there is a need to clone using SSH, configure the SSH
authorization under the \texttt{clone-auth} element.  This element can contain the following
key:value pair combinations:

\noindent\\\textbf{Token Authorization Elements}

\noindent\\To access the SCM API or clone with a token, the following elements can appear under the 
\texttt{api-auth} element exclusive of other elements:

\begin{itemize}
    \item \texttt{token} - The value specifies a file name found under the path defined
    by \texttt{secret-root-path} containing a Personal Access Token (PAT).  This is required for
    token authorization.
    \item \texttt{username} - The value specifies a file name found under the path defined
    by \texttt{secret-root-path} containing a username associated with the PAT.  This is 
    optional and only used during cloning; if not supplied, the default username of \texttt{git} is used.
\end{itemize}

\noindent\\\textbf{Basic Authorization Elements}

\noindent\\To access the SCM API or clone with basic authorization, the following required elements can
appear under the \texttt{api-auth} element exclusive of other elements:

\begin{itemize}
    \item \texttt{username} - The value specifies a file name found under the path defined
    by \texttt{secret-root-path} containing the username associated with the account used
    for authorization. 
    \item \texttt{password} - The value specifies a file name found under the path defined
    by \texttt{secret-root-path} containing the password associated with the account used
    for authorization. 
\end{itemize}




