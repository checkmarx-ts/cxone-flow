\chapter{Configuration}


\section{Runtime Configuration}\label{sec:runtime-config}

\subsection{SSL}

\subsubsection{Trusting Self-Signed Certificates}\label{sec:self-signed-certs}

While the \cxone system uses TLS certificates signed by a public CA, it is possible that
corporate proxies use certificates signed by a private CA. If so, it is possible to
import custom CA certificates when using \cxoneflow.

\noindent\\Each custom certificate to import must meet the following criteria:

\begin{itemize}
    \item Must be in a file ending with the extension .crt.
    \item The contents of the file must be one certificate stored in the PEM format.
    \item Each file must be mapped to the container path /usr/local/share/ca-certificates.
\end{itemize}


\noindent\\As an example, if using Docker, it is possible to map a single local file to a file in the container with this mapping 
option added to the container execution command line:

\begin{code}{Custom CA Mapping Option}{[Docker]}{}
-v $(pwd)/custom-ca.pem:/usr/local/share/ca-certificates/custom-ca.crt
\end{code}

\subsubsection{The \texttt{ssl-verify} Option}\label{sec:ssl-verify-general}

In the configuration YAML documentation, all of the \texttt{connection}
elements contain an optional \texttt{ssl-verify} setting.  This option
is generally useful to turn off SSL verification by setting it to \texttt{False}.
This can also be used to control which CA bundle is used for verification.

\noindent\\Omitting the \texttt{ssl-verify} setting should be sufficient for
most deployment cases.  If omitted, the container execution will use the default CA bundle
where any custom CAs are added as described in Section \ref{sec:self-signed-certs}.
The \texttt{ssl-verify} option can be set to an explicit path on the container
if there is a need to use a CA bundle other than the one provided by the OS.


\subsubsection{Configuring SSL for the \cxoneflow Endpoint}

To make the \cxoneflow endpoint use SSL for communication, obtain an SSL certificate public/private key pair
and map the files to a location on the container.  The following environment variables must then be set in the
runtime environment:

\begin{table}[ht]
    \caption{SSL Environment Variables}        
    \begin{tabularx}{\textwidth}{ll}
        \toprule
        \textbf{Variable} & \textbf{Description}\\
        \midrule
        \texttt{SSL\_CERT\_PATH} & \makecell[l]{The path to the server's SSL certificate in PEM format.}\\
        \midrule
        \texttt{SSL\_CERT\_KEY\_PATH} & \makecell[l]{The path to the certificate's unencrypted private key.}\\
        \bottomrule
    \end{tabularx}
\end{table}

\noindent\\If your SSL certificate is self-signed or signed with a non-public CA, you'll want
to import the self-signed certificate and/or non-public CA as described in Section \ref{sec:self-signed-certs}.


\subsection{Runtime Control Environment Variables}

\begin{table}[ht]
    \caption{Runtime Control Environment Variables}        
    \begin{tabularx}{\textwidth}{lccl}
        \toprule
        \textbf{Variable} & \textbf{Required} & \textbf{Default} & \textbf{Description}\\
        \midrule
        \texttt{CXONEFLOW\_WORKERS} & No & \texttt{max(\# of CPUs / 2, 1)} & \makecell[l]{The number of worker processes\\used for parallel execution. The\\maximum value will be\\set at \texttt{(\# of CPUs - 1)}}\\
        \midrule
        \texttt{LOG\_LEVEL} & No & \texttt{INFO} & \makecell[l]{The logging verbosity level.  Set to\\\texttt{DEBUG} for increased logging\\verbosity.}\\
        \midrule
        \texttt{CONFIG\_YAML\_PATH} & No & \texttt{/opt/cxone/config.yaml} & \makecell[l]{The path to the configuration\\YAML file.}\\
        \midrule
        \texttt{CXONEFLOW\_HOSTNAME} & No & \texttt{localhost} & \makecell[l]{The virtual hostname of the\\\cxoneflow endpoint.}\\
        \bottomrule
    \end{tabularx}
\end{table}


\newpage

\section{Operational Configuration}\label{sec:op-config}

The operational configuration uses a YAML file mapped at \texttt{/opt/cxone/config.yaml}
by default.  It is possible to map the \texttt{config.yaml} file to another location in the
container and adjust the path via the \texttt{CONFIG\_YAML\_PATH} environment variable.

\subsection{YAML Configuration Examples}

\subsubsection{Basic YAML Configurations}

\input{operation/yaml_minimal_example.tex}

\newpage
\noindent\\\input{operation/yaml_full_example.tex}


\subsubsection{Complex YAML Configurations using YAML Anchors}

For complex configurations, it is possible to use 
\href{https://docs.docker.com/compose/compose-file/10-fragments/}{YAML Anchors}
to avoid repeating some section definitions.  When using YAML anchors, it may be useful
to use a \href{https://onlineyamltools.com/convert-yaml-to-json}{YAML-to-JSON} conversion tool that shows the JSON generated from the YAML
definition

\noindent\\\input{operation/yaml_anchors_example.tex}

\subsection{YAML Configuration Elements}\label{sec:yaml-config}

The organization of the YAML configuration is depicted in the tree below.  The description of each element
can be referenced by clicking the element.  Required elements are indicated in the tree; in general, if an
element that is not marked "required" is omitted, the feature that performs that operation is not invoked
for the configured service definition.

\paragraph{YAML Root Elements}

\noindent\\

\dirtree{%
    .1 \hyperref[sec:yaml-root]{<root>}.
    .2 \hyperref[sec:yaml-secret-root-path]{secret-root-path} \DTcomment{[Required]}.
    .2 \hyperref[sec:yaml-server-base-url]{server-base-url} \DTcomment{[Required]}.
    .2 \hyperref[sec:yaml-scm-monikers]{<scm moniker>} \DTcomment{[Required: \textbf{bbdc}, \textbf{adoe}, \textbf{gh}]}.
    .3 \hyperref[sec:moniker-elements]{...see "YAML SCM Moniker Elements"}.
}

\pagebreak
\paragraph{YAML SCM Moniker Elements}\label{sec:moniker-elements}
\noindent\\

\dirtree{%
    .1 \hyperref[sec:yaml-root]{<root>}.
    .2 \hyperref[sec:yaml-scm-monikers]{<scm moniker>} \DTcomment{[Required: \textbf{bbdc}, \textbf{adoe}, \textbf{gh}]}.
    .3 \hyperref[sec:yaml-moniker-connection]{connection} \DTcomment{[Required]}.
    .4 \hyperref[sec:connection-elements]{...see "YAML \texttt{connection} Elements"}.
    .3 \hyperref[sec:yaml-moniker-cxone]{cxone} \DTcomment{[Required]}.
    .4 \hyperref[sec:cxone-elements]{...see "YAML \texttt{cxone} Elements"}.
    .3 \hyperref[sec:yaml-moniker-feedback]{feedback} \DTcomment{[Optional]}.
    .4 \hyperref[sec:feedback-elements]{...see "YAML \texttt{feedback} Elements"}.
    .3 \hyperref[sec:yaml-moniker-repo-match]{repo-match} \DTcomment{[Required]}.
    .3 \hyperref[sec:yaml-moniker-resolver]{resolver} \DTcomment{[Optional]}.
    .4 \hyperref[sec:resolver-elements]{...see "YAML \texttt{resolver} Elements"}.
    .3 \hyperref[sec:yaml-moniker-scan-config]{scan-config} \DTcomment{[Optional]}.
    .4 \hyperref[sec:scan-config-elements]{...see "YAML \texttt{scan-config} Elements"}.
    .3 \hyperref[sec:yaml-moniker-service-name]{service-name} \DTcomment{[Required]}.
}

\pagebreak
\paragraph{YAML \texttt{connection} Elements}\label{sec:connection-elements}
\noindent\\

\dirtree{%
    .1 \hyperref[sec:yaml-root]{<root>}.
    .2 \hyperref[sec:yaml-scm-monikers]{<scm moniker>} \DTcomment{[Required: \textbf{bbdc}, \textbf{adoe}, \textbf{gh}]}.
    .3 \hyperref[sec:yaml-moniker-connection]{connection} \DTcomment{[Required]}.
    .4 \hyperref[sec:yaml-connection-base-url]{base-url} \DTcomment{[Required]}.
    .4 \hyperref[sec:yaml-connection-base-display-url]{base-display-url} \DTcomment{[Required for some SCMs]}.
    .4 \hyperref[sec:yaml-connection-api-url-suffix]{api-url-suffix} \DTcomment{[Required for some SCMs]}.
    .4 \hyperref[sec:yaml-connection-shared-secret]{shared-secret} \DTcomment{[Required]}.
    .4 \hyperref[sec:yaml-generic-proxies]{proxies} \DTcomment{[Optional]}.
    .4 \hyperref[sec:yaml-generic-retries]{retries} \DTcomment{[Optional] Default: 3}.
    .4 \hyperref[sec:yaml-generic-ssl-verify]{ssl-verify} \DTcomment{[Optional] Default: True}.
    .4 \hyperref[sec:yaml-generic-timeout-seconds]{timeout-seconds}\DTcomment{[Optional] Default: 60s}.
    .4 \hyperref[sec:yaml-connection-api-auth]{api-auth} \DTcomment{[Required]}.
    .5 \hyperref[sec:yaml-api-auth-app-private-key]{app-private-key} \DTcomment{[See element documentation]}.
    .5 \hyperref[sec:yaml-api-auth-password]{password} \DTcomment{[See element documentation]}.
    .5 \hyperref[sec:yaml-api-auth-token]{token} \DTcomment{[See element documentation]}.
    .5 \hyperref[sec:yaml-api-auth-username]{username} \DTcomment{[See element documentation]}.
    .4 \hyperref[sec:yaml-connection-clone-auth]{clone-auth} \DTcomment{[Optional] Default: \texttt{api-auth}}.
    .5 \hyperref[sec:yaml-api-auth-password]{password} \DTcomment{[See element documentation]}.
    .5 \hyperref[sec:yaml-clone-auth-ssh]{ssh} \DTcomment{[See element documentation]}.
    .5 \hyperref[sec:yaml-clone-auth-ssh-port]{ssh-port} \DTcomment{[See element documentation]}.
    .5 \hyperref[sec:yaml-api-auth-token]{token} \DTcomment{[See element documentation]}.
    .5 \hyperref[sec:yaml-api-auth-username]{username} \DTcomment{[See element documentation]}.
}

\pagebreak
\paragraph{YAML \texttt{cxone} Elements}\label{sec:cxone-elements}
\noindent\\

\dirtree{%
    .1 \hyperref[sec:yaml-root]{<root>}.
    .2 \hyperref[sec:yaml-scm-monikers]{<scm moniker>} \DTcomment{[Required: \textbf{bbdc}, \textbf{adoe}, \textbf{gh}]}.
    .3 \hyperref[sec:yaml-moniker-cxone]{cxone} \DTcomment{[Required]}.
    .4 \hyperref[sec:yaml-cxone-api-endpoint]{api-endpoint} \DTcomment{[Required]}.
    .4 \hyperref[sec:yaml-cxone-api-key]{api-key} \DTcomment{[Required without oauth]}.
    .4 \hyperref[sec:yaml-cxone-iam-endpoint]{iam-endpoint} \DTcomment{[Required]}.
    .4 \hyperref[sec:yaml-cxone-oauth]{oauth} \DTcomment{[Required without api-key]}.
    .4 \hyperref[sec:yaml-generic-proxies]{proxies} \DTcomment{[Optional]}.
    .4 \hyperref[sec:yaml-generic-retries]{retries} \DTcomment{[Optional] Default: 3}.
    .4 \hyperref[sec:yaml-generic-ssl-verify]{ssl-verify} \DTcomment{[Optional] Default: True}.
    .4 \hyperref[sec:yaml-cxone-tenant]{tenant} \DTcomment{[Required]}.
    .4 \hyperref[sec:yaml-generic-timeout-seconds]{timeout-seconds}\DTcomment{[Optional] Default: 60s}.
}


\pagebreak
\paragraph{YAML \texttt{feedback} Elements}\label{sec:feedback-elements}
\noindent\\


\dirtree{%
    .1 \hyperref[sec:yaml-root]{<root>}.
    .2 \hyperref[sec:yaml-scm-monikers]{<scm moniker>} \DTcomment{[Required: \textbf{bbdc}, \textbf{adoe}, \textbf{gh}]}.
    .3 \hyperref[sec:yaml-moniker-feedback]{feedback} \DTcomment{[Optional]}.
    .4 \hyperref[sec:yaml-generic-amqp]{amqp} \DTcomment{[Optional] Default: container instance}.
    .5 \hyperref[sec:yaml-generic-amqp-amqp-password]{amqp-password} \DTcomment{[Optional]}.
    .5 \hyperref[sec:yaml-generic-amqp-amqp-url]{amqp-url} \DTcomment{[Required]}.
    .5 \hyperref[sec:yaml-generic-amqp-amqp-user]{amqp-user} \DTcomment{[Optional]}.
    .5 \hyperref[sec:yaml-generic-ssl-verify]{ssl-verify} \DTcomment{[Optional] Default: True}.
    .4 \hyperref[sec:yaml-feedback-pull-request]{pull-request} \DTcomment{[Optional]}.
    .5 \hyperref[sec:yaml-pull-request-enabled]{enabled} \DTcomment{[Optional] Default: False}.
    .4 \hyperref[sec:yaml-feedback-scan-monitor]{scan-monitor} \DTcomment{[Optional]}.
    .5 \hyperref[sec:yaml-scan-monitor-poll-backoff-multiplier]{poll-backoff-multiplier} \DTcomment{[Optional] Default: 2}.
    .5 \hyperref[sec:yaml-scan-monitor-poll-interval-seconds]{poll-interval-seconds} \DTcomment{[Optional] Default: 90s}.
    .5 \hyperref[sec:yaml-scan-monitor-poll-max-interval-seconds]{poll-max-interval-seconds} \DTcomment{[Optional] Default: 600s}.
    .5 \hyperref[sec:yaml-scan-monitor-scan-timeout-hours]{scan-timeout-hours} \DTcomment{[Optional] Default: 48h}.
    .4 \hyperref[sec:yaml-feedback-exclusions]{exclusions} \DTcomment{[Optional]}.
    .5 \hyperref[sec:yaml-exclusions-severity]{severity} \DTcomment{[Optional]}.
    .5 \hyperref[sec:yaml-exclusions-state]{state} \DTcomment{[Optional]}.
}


\pagebreak
\paragraph{YAML \texttt{resolver} Elements}\label{sec:resolver-elements}
\noindent\\

\dirtree{%
    .1 \hyperref[sec:yaml-root]{<root>}.
    .2 \hyperref[sec:yaml-scm-monikers]{<scm moniker>} \DTcomment{[Required: \textbf{bbdc}, \textbf{adoe}, \textbf{gh}]}.
    .3 \hyperref[sec:yaml-moniker-resolver]{resolver} \DTcomment{[Optional]}.
}


\pagebreak
\paragraph{YAML \texttt{scan-config} Elements}\label{sec:scan-config-elements}
\noindent\\

\dirtree{%
    .1 \hyperref[sec:yaml-root]{<root>}.
    .2 \hyperref[sec:yaml-scm-monikers]{<scm moniker>} \DTcomment{[Required: \textbf{bbdc}, \textbf{adoe}, \textbf{gh}]}.
    .3 \hyperref[sec:yaml-moniker-scan-config]{scan-config} \DTcomment{[Optional]}.
    .4 \hyperref[sec:yaml-scan-config-default-scan-engines]{default-scan-engines} \DTcomment{[Optional]}.
    .4 \hyperref[sec:yaml-scan-config-default-project-tags]{default-project-tags} \DTcomment{[Optional]}.
    .4 \hyperref[sec:yaml-scan-config-default-scan-tags]{default-scan-tags} \DTcomment{[Optional]}.
}


\subsubsection{YAML Element: Root}\label{sec:yaml-root}

The root elements of the YAML configuration are formatted to the left-most
position in the YAML file.  Anchor elements may be defined at the root but
must not clash with the names of any of the root elements.

\subsubsection{YAML Element: secret-root-path}\label{sec:yaml-secret-root-path}

A string that is the path to a directory that contains one or more files containing secret values.  The names to these files are 
referenced elsewhere in the YAML configuration file when used in a field that is a reference to a secret.

\subsubsection{YAML Element: server-base-url}\label{sec:yaml-server-base-url}
A string that is the base URL for the \cxoneflow endpoint.  This is used when creating feedback content that loads image elements.

\subsubsection{YAML Element: <scm moniker>}\label{sec:yaml-scm-monikers}

This is a moniker indicating the a list of service definitions for handling events from an SCM matching the name of the SCM
moniker.  Each service definition is a YAML dictionary of elements. The contents for each service definition dictionary 
have the same meaning unless otherwise specified.  At lease one SCM moniker with one configured service definition is required. 
The following SCM configuration monikers are currently supported:

\begin{itemize}
    \item \textbf{\texttt{bbdc}} for BitBucket Data Center webhook payloads targeting the \texttt{/bbdc}
    webhook payload receiver endpoint.
    \item \textbf{\texttt{adoe}} for Azure DevOps Enterprise or Cloud webhook payloads targeting the \texttt{/adoe}
    webhook payload receiver endpoint.
    \item \textbf{\texttt{gh}} for GitHub Enterprise or Cloud webhook payloads targeting the \texttt{/gh}
    webhook payload receiver endpoint.
\end{itemize}


\input{operation/yaml/scm-monikers.tex}
\input{operation/yaml/scan-config.tex}
\input{operation/yaml/cxone.tex}
\input{operation/yaml/feedback.tex}
\input{operation/yaml/pull-request.tex}
\input{operation/yaml/scan-monitor.tex}
\input{operation/yaml/exclusions.tex}
\input{operation/yaml/connection.tex}
\input{operation/yaml/api-auth.tex}
\input{operation/yaml/clone-auth.tex}
\input{operation/yaml/generic.tex}







