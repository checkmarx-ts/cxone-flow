The following example shows a minimal \cxoneflow configuration that defines the following:

\begin{enumerate}
    \item Files containing secrets are located at \texttt{/run/secrets}.
    \item One BitBucket Data Center SCM connection configuration to handle all webhook payloads
    POSTed to the \texttt{/bbdc} endpoint.
    \item One catch-all route for BitBucket clone-urls using the regular expression \texttt{.*}
    \item The BitBucket base URL located at \texttt{https://scm.corp.com}
    \item The shared secret used to validate webhook payloads located in the file \texttt{/run/secrets/scm-shared-secret}
    \item The API and clone authorization using a PAT in a file located at \texttt{/run/secrets/scm-token-secret}
    \item The \cxone tenant name of \texttt{mytenant}
    \item The \cxone API credentials using an \texttt{oauth} client with:
    \begin{enumerate}
        \item The client identifier located in the file \texttt{/run/secrets/my-oauth-id}
        \item The client secret located in the file \texttt{/run/secrets/my-oauth-secret}
    \end{enumerate}
    \item Using the \cxone multi-tenant US region IAM endpoint.
    \item Using the \cxone multi-tenant US region API endpoint.
\end{enumerate}

\begin{code}{Minimal YAML Configuration Example \#1}{Using BitBucket Data Center}{}
secret-root-path: /run/secrets
server-base-url: https://cxoneflow.mydomain.com:8443/

bbdc:
    - service-name: BitBucket DC
      repo-match: ^http(s)?:(\/){2}bitbucket\.corp\.com.*
      connection:
        base-url: https://bitbucket.corp.com
        shared-secret: scm-shared-secret
        api-auth:
          token: scm-token-secret
      cxone:
        tenant: mytenant
        oauth:
          client-id: my-oauth-id
          client-secret: my-oauth-secret
        iam-endpoint: US
        api-endpoint: US
\end{code}

\pagebreak
An alternate minimal example using a token for API authorization and SSH for cloning authorization:

\begin{code}{Minimal YAML Configuration Example \#2}{Using BitBucket Data Center}{}
secret-root-path: /run/secrets
server-base-url: https://cxoneflow.mydomain.com:8443/

bbdc:
    - service-name: BitBucket DC
      repo-match: ^http(s)?:(\/){2}bitbucket\.corp\.com.*
      connection:
      base-url: https://bitbucket.corp.com
      shared-secret: scm-shared-secret
      api-auth:
        token: scm-token-secret
      cxone:
        tenant: mytenant
        api-key: my-cxone-api-key
        iam-endpoint: US
        api-endpoint: US
\end{code}
    
\pagebreak
An alternate minimal example using for an Azure DevOps Enterprise SCM:

\begin{code}{Minimal YAML Configuration Example \#3}{Using Azure DevOps Enterprise}{}
secret-root-path: /run/secrets
server-base-url: https://cxoneflow.mydomain.com:8443/

adoe:
    - service-name: MyADO
      repo-match: ^http(s)?:(\/){2}ado\.corp\.com.*
      connection:
      base-url: https://ado.corp.com/
      shared-secret: scm-shared-secret
      api-auth:
        username: scm-username-secret
        password: scm-password-secret
      cxone:
        tenant: mytenant
        api-key: my-cxone-api-key
        iam-endpoint: US
        api-endpoint: US
\end{code}

\pagebreak
A minimal example for BitBucket Data Center with pull-request feedback enabled:

\begin{code}{Minimal YAML Configuration Example}{Enabling Pull Request Feedback}{}
secret-root-path: /run/secrets
server-base-url: https://cxoneflow.mydomain.com:8443/

bbdc:
    - service-name: BitBucket DC
      repo-match: ^http(s)?:(\/){2}bitbucket\.corp\.com.*
      feedback:
        pull-request:
          enabled: True
      connection:
        base-url: https://bitbucket.corp.com
        shared-secret: scm-shared-secret
        api-auth:
          token: scm-token-secret
      cxone:
        tenant: mytenant
        oauth:
          client-id: my-oauth-id
          client-secret: my-oauth-secret
        iam-endpoint: US
        api-endpoint: US
\end{code}

\pagebreak
A minimal example using GitHub Enterprise with the \cxoneflow integration installed
as a GitHub App:
\begin{code}{Minimal YAML Configuration Example}{GitHub Enterprise as GitHub App}{}
secret-root-path: /run/secrets
server-base-url: https://cxoneflow.mydomain.com:8443/

gh:
    - service-name: GHE
      repo-match: .*
      feedback:
        pull-request:
          enabled: True
      connection:
        base-url: https://scm.corp.com
        api-url-suffix: api/v3
        shared-secret: scm-shared-secret
        api-auth:
          app-private-key: ghe-app-priv-key
      cxone:
        tenant: mytenant
        oauth:
          client-id: my-oauth-id
          client-secret: my-oauth-secret
        iam-endpoint: US
        api-endpoint: US
\end{code}

  
  
\pagebreak
A minimal example using GitHub Enterprise with the \cxoneflow integration using
webhooks configured at the individual repository scope:
\begin{code}{Minimal YAML Configuration Example}{GitHub Enterprise with Webhooks}{}
secret-root-path: /run/secrets
server-base-url: https://cxoneflow.mydomain.com:8443/

gh:
    - service-name: GHE
      repo-match: ^http(s)?:(\/){2}github\.corp\.com.*
      feedback:
        pull-request:
          enabled: True
      connection:
        base-url: https://github.corp.com
        api-url-suffix: api/v3
        shared-secret: scm-shared-secret
        api-auth:
          token: ghe-token
      cxone:
        tenant: mytenant
        oauth:
          client-id: my-oauth-id
          client-secret: my-oauth-secret
        iam-endpoint: US
        api-endpoint: US
\end{code}
  




\pagebreak
A minimal example using GitHub Cloud with the \cxoneflow integration installed
as a GitHub App:
\begin{code}{Minimal YAML Configuration Example}{GitHub Cloud as GitHub App}{}
secret-root-path: /run/secrets
server-base-url: https://cxoneflow.mydomain.com:8443/

gh:
    - service-name: GHC
      repo-match: .*
      feedback:
        pull-request:
          enabled: True
      connection:
        base-url: https://api.github.com
        base-display-url: https://www.github.com
        shared-secret: scm-shared-secret
        api-auth:
          app-private-key: ghc-app-priv-key
      cxone:
        tenant: mytenant
        oauth:
          client-id: my-oauth-id
          client-secret: my-oauth-secret
        iam-endpoint: US
        api-endpoint: US
\end{code}
  

  
  
\pagebreak
A minimal example using GitHub Cloud with the \cxoneflow integration using
webhooks configured at the individual repository level:
\begin{code}{Minimal YAML Configuration Example}{GitHub Cloud with Webhooks}{}
secret-root-path: /run/secrets
server-base-url: https://cxoneflow.mydomain.com:8443/

gh:
    - service-name: GHE
        repo-match: .*
        feedback:
        pull-request:
            enabled: True
        connection:
        base-url: https://api.github.com
        base-display-url: https://www.github.com
        shared-secret: scm-shared-secret
        api-auth:
            token: ghc-token
        cxone:
        tenant: mytenant
        oauth:
            client-id: my-oauth-id
            client-secret: my-oauth-secret
        iam-endpoint: US
        api-endpoint: US
\end{code}
  

\pagebreak
A minimal example using Gitlab with the \cxoneflow integration using
webhooks configured at any scope:
\begin{code}{Minimal YAML Configuration Example}{Gitlab Webhooks}{}
  secret-root-path: /run/secrets
  server-base-url: https://cxoneflow.mydomain.com:8443/
  
  gh:
      - service-name: Gitlab
        repo-match: ^http(s)?:(\/){2}gitlab\.corp\.com.*
        feedback:
          pull-request:
            enabled: True
        connection:
          base-url: https://gitlab.corp.com
          api-url-suffix: api/v4
          shared-secret: scm-shared-secret
          api-auth:
            token: ghe-token
        cxone:
          tenant: mytenant
          oauth:
            client-id: my-oauth-id
            client-secret: my-oauth-secret
          iam-endpoint: US
          api-endpoint: US
\end{code}
    
