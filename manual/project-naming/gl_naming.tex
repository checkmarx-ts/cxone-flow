The dynamic project naming script in the code listing for the \texttt{gl\_naming}
module is shown below.  This changes the naming convention of the projects to create
a path containing only the upper-case characters in the namespace path and to use
the project's full name at the end of the abbreviated namespace path.

The example utilizes the SCM's API to get the project details that are not included
with the webhook event payload.\\

\begin{code}{Gitlab Dynamic Project Naming Module}{[ghe\_naming.py]}{}
from api_utils.auth_factories import EventContext
from scm_services import BasicSCMService
import logging

async def event_project_name_factory(context: EventContext, scm_service: BasicSCMService) -> str:
    # Get an instance of the logger
    log = logging.getLogger(__name__)

    # Output the event context to the debug log
    log.debug(context)

    if "project" in context.message.keys():
        project_name = context.message["project"]["name"]

        # Get the full proper name of the repository from the project configuration.
        resp = await scm_service.exec("GET", f"/projects/{context.message['project']['id']}")
        if resp.ok:
            full_name = resp.json()["name_with_namespace"]

            path_components = full_name.split("/")

            # Use all components except the last one which is the name of the project.
            revised_components = []
            for component in path_components[:-1]:
                revised_components.append("".join(c for c in component if c.isupper()))
            
            revised_components.append(project_name)

            return "/".join(revised_components)


    # Failure causes the default project name to be used
    return None

\end{code}
