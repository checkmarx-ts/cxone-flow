\chapter{Distributed Resolver Agents}\label{sec:resolver-agents}

\section{Overview}

Executing Software Composition Analysis (SCA) scans in CheckmarxOne will gather results
by analyzing packages found in the dependency tree for the code submitted for scan.  The
dependency analysis for the code, when submitted to CheckmarxOne, is performed
in the CheckmarxOne environment by executing the package managers that match the composition of the code
under scan.  This generally works sufficiently when the code references only publicly-available, open-source
packages and is compatible with the package manager tooling installed in the CheckmarxOne environment.  The
CheckmarxOne SCA scan first obtains a dependency tree by executing the package manager tools, then performs an analysis
of the packages found in the dependency tree.

Not all software is composed of only publicly available packages nor does it always use package manager tooling
that is compatible with those installed in the CheckmarxOne environment.  Incompatibilities usually manifest
when some of the following issues are observed:

\begin{itemize}
  \item The dependency tree is incomplete when software references a private package repository that has
  no network connection path available for connections from the CheckmarxOne server.
  \item The dependency tree is incomplete when the code under scan is incompatible with the 
  package manager tools installed in the CheckmarxOne environment.
\end{itemize}

The general solution to obtain an accurate dependency tree when it is not possible to obtain 
via package manager tool execution in the CheckmarxOne environment is to perform the dependency resolution using the 
\href{https://docs.checkmarx.com/en/34965-19197-checkmarx-sca-resolver-download-and-installation.html}{SCA Resolver}.
The SCA Resolver is typically scripted to execute in a pipeline prior to the scan submission to CheckmarxOne.  This performs the dependency resolution
in the same environment as the code builds, which will generally resolve any tooling compatibility or network connection
problems.  

Since \cxoneflow is primarily driven by asynchronous web hook events and does not invoke a pipeline
where dependency resolution can be scripted, the distributed resolver agents are used to perform the dependency resolution using
\href{https://docs.checkmarx.com/en/34965-19197-checkmarx-sca-resolver-download-and-installation.html}{SCA Resolver}.
The diagram in Figure \ref{fig:resolver-agent-diagram} shows a typical deployment of the resolver agent.  

The resolver agent is deployed in the same build environment where the code typically builds in a CI/CD pipeline using a
self-hosted CI/CD "runner" agent. The \cxoneflow server delegates SCA Resolver to execute using methods similar to how a
CI/CD pipeline delegates build script
execution via the self-hosted "runner" agent.  The package manager tools used to produce the dependency tree match the versions, configurations
and network connection capabilities found on the self-hosted runner.  The resolver agent can execute the dependency resolution in a shell
that references tools installed on the runner machine, or can optionally execute tooling in a build environment defined in a container.


\begin{figure}[ht]
  \includegraphics[width=\textwidth]{graphics/cxoneflow-diagrams-Resolver Agent Diagram.png}
  \caption{Resolver Agent Deployment Diagram}
  \label{fig:resolver-agent-diagram}
\end{figure}


\section{Server Configuration}\label{sec:resolver-server}

\subsection{\cxoneflow Endpoint Configuration}
TBD


\subsection{Message Queue Configuration}

TBD
\section{Resolver Scan Agent Configuration}\label{sec:resolver-agent}

\subsection{Overview}





\subsection{Security Considerations}


\subsection{Installation and Configuration}

\subsubsection{Debian Linux Platforms}

securing tokens/MQ creds in resolver agent

securing runtime activities of the resolver agent


\subsection{Limitations}
SSH clone does not work with resolver agent scanning



