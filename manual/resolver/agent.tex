\section{Resolver Scan Agent Configuration}\label{sec:resolver-agent}

\subsection{Overview}


\subsection{Security Considerations}\label{sec:resolver-agent-security}

As briefly described in Section \ref{sec:dist-resolver-security-considerations}, executing
part of a code build to extract a dependency tree can pose some risks. For example,
tools like Gradle define the build definition in the programming language Groovy; this Groovy
code executes when Gradle loads the build definition.  Other tools even execute scripts defined
in the dependency package leading to the popular method of "typo-squatting" as a means to
perform a remote-code-execution (RCE) attack directly on unsuspecting developers.  Executing an
untrusted build definition or loading a dependency containing malware into a build system is
not desirable.

Since the risk of malicious builds is always possible, the deployment of the distributed resolver agent
can be configured to mitigate these risks.  The section for each platform's install makes specific
recommendations for a secure deployment of the distributed resolver agent.  The general recommendations
for all platform deployments are:

\begin{itemize}
  \item Utilize file permissions on the secret values used by the agent to limit which running processes
    can read the contents of the secrets
  \item Use the recommended directory permissions for each platform's deployment.
  \item Find a logical grouping of agents and use different message queue credentials to better avoid
    impact if any credentials are misused.
  \item Use the account security settings to limit message queue access to agents such that they are only
    able to receive and send events required for their specific operations.
  \item If the agent is invoking \scaresolver as a shell execution, utilize a "run-as" configuration to
    run the resolver as a low-privilege account.
  \item Utilize the \scaresolver docker execution capability to sandbox the dependency resolution in a container.
  \item If configuring an agent to execute \scaresolver for both shell and container dependency resolution,
    configure your docker daemon to run in "rootless" mode.
\end{itemize}


\subsubsection{Message Queue Authorization}

securing tokens/MQ creds in resolver agent


\subsubsection{Shell Agent Isolation}

securing runtime activities of the resolver agent

\paragraph{Linux}

\paragraph{Windows}
\noindent\\Distributed resolver agents are not currently supported on Windows.


\subsubsection{Container Agent Isolation}

securing runtime activities of the resolver agent

\paragraph{Linux}

\paragraph{Windows}
\noindent\\Using container agents are not supported on Windows.


\subsection{Installation}
pre-requisites:
installer for platform
message queue credentials and connection information
a public key that matches the private key deployed on the endpoint

\subsubsection{Debian Linux Platforms}

\subsection{Configuration}

\subsection{Limitations}
SSH clone does not work with resolver agent scanning


\subsection{Distributed Resolver Agent YAML Configuration}

The distributed resolver agent's YAML configuration is deployed after the agent is installed.
Many of the elements share format with the \hyperref[sec:yaml-config]{server's configuration elements}. 


Below is an example of a distributed agent YAML configuration.  This example demonstrates:

\begin{itemize}
  \item Secrets are stored in \texttt{/var/secrets}
  \item A common configuration block used by each tag is placed in each tag's configuration using YAML Anchors.
  \item \scaresolver execution uses a work path of \texttt{/var/resolver}.
  \item Shell execution of \scaresolver uses an instance installed at \texttt{/opt/resolver/ScaResolver}.
  \item Options for \scaresolver are defined in the \texttt{resolver-opts} element.
  \item All agent tags use the same AMQP configuration via YAML Anchor tags.
  \item The agent tag \texttt{general} is configured to execute \scaresolver as a shell execution.
  \item The agent tag \texttt{java-gradle} executes \scaresolver using a container image \texttt{gradle:8-jdk17-alpine} that
    is then extended by the agent using the \toolkit installed at\\\texttt{/opt/supply-chain-build-env}.
\end{itemize}

\input{resolver/yaml/example.tex}


\pagebreak

\subsubsection{Distributed Resolver Agent YAML Configuration Tree}\label{sec:agent-yaml-root}

\dirtree{%
    .1 \hyperref[sec:agent-yaml-root]{<root>}.
    .2 \hyperref[sec:yaml-secret-root-path]{secret-root-path}\DTcomment{[Required]}.
    .2 \hyperref[sec:agent-serviced-tags]{serviced-tags}\DTcomment{[Required]}.
    .3 \hyperref[sec:agent-tag]{<agent tag>}\DTcomment{[At least 1 required]}.
    .4 \hyperref[sec:yaml-generic-amqp]{amqp}\DTcomment{[Optional] Default: localhost}.
    .5 \hyperref[sec:yaml-generic-amqp-amqp-password]{amqp-password}\DTcomment{[Optional]}.
    .5 \hyperref[sec:yaml-generic-amqp-amqp-url]{amqp-url}\DTcomment{[Required]}.
    .5 \hyperref[sec:yaml-generic-amqp-amqp-user]{amqp-user}\DTcomment{[Optional]}.
    .5 \hyperref[sec:yaml-generic-ssl-verify]{ssl-verify}\DTcomment{[Optional] Default: True}.
    .4 \hyperref[sec:agent-public-key]{public-key}\DTcomment{[Required]}.
    .4 \hyperref[sec:agent-resolver-opts]{resolver-opts}\DTcomment{[Optional] Default: None}.
    .4 \hyperref[sec:agent-resolver-path]{resolver-path}\DTcomment{[Required if not \texttt{run-with-container}]}.
    .4 \hyperref[sec:agent-resolver-run-as]{resolver-run-as}\DTcomment{[Optional] Default: run as service user}.
    .4 \hyperref[sec:agent-resolver-work-path]{resolver-work-path}\DTcomment{[Optional] Default: /tmp/resolver}.
    .4 \hyperref[sec:agent-run-with-container]{run-with-container}\DTcomment{[Required if not \texttt{resolver-path}]}.
    .5 \hyperref[sec:agent-container-image-tag]{container-image-tag}\DTcomment{[Required]}.
    .5 \hyperref[sec:agent-supply-chain-toolkit-path]{supply-chain-toolkit-path}\DTcomment{[Required]}.
    .5 \hyperref[sec:agent-use-running]{use-running-gid}\DTcomment{[Optional] Default: True}.
    .5 \hyperref[sec:agent-use-running]{use-running-uid}\DTcomment{[Optional] Default: True}.
}

\subsubsection{YAML Element: Root}\label{sec:yaml-root}

The root elements of the YAML configuration are formatted to the left-most
position in the YAML file.  Anchor elements may be defined at the root but
must not clash with the names of any of the root elements.

\subsubsection{YAML Element: secret-root-path}\label{sec:yaml-secret-root-path}

A string that is the path to a directory that contains one or more files containing secret values.  The names to these files are 
referenced elsewhere in the YAML configuration file when used in a field that is a reference to a secret.

\subsubsection{YAML Element: server-base-url}\label{sec:yaml-server-base-url}
A string that is the base URL for the \cxoneflow endpoint.  This is used when creating feedback content that loads image elements.

\subsubsection{YAML Element: <scm moniker>}\label{sec:yaml-scm-monikers}

This is a moniker indicating the a list of service definitions for handling events from an SCM matching the name of the SCM
moniker.  Each service definition is a YAML dictionary of elements. The contents for each service definition dictionary 
have the same meaning unless otherwise specified.  At lease one SCM moniker with one configured service definition is required. 
The following SCM configuration monikers are currently supported:

\begin{itemize}
    \item \textbf{\texttt{bbdc}} for BitBucket Data Center webhook payloads targeting the \texttt{/bbdc}
    webhook payload receiver endpoint.
    \item \textbf{\texttt{adoe}} for Azure DevOps Enterprise or Cloud webhook payloads targeting the \texttt{/adoe}
    webhook payload receiver endpoint.
    \item \textbf{\texttt{gh}} for GitHub Enterprise or Cloud webhook payloads targeting the \texttt{/gh}
    webhook payload receiver endpoint.
\end{itemize}


\input{resolver/yaml/serviced-tags.tex}
\input{resolver/yaml/tag.tex}

