\chapter{Common Workflow Elements}

Each source control system will use common Git concepts such as branching
and pull-requests.  Scanning workflows are orchestrated based on these common concepts as
described in Section \ref{sec:overview}.  

Some source control systems will have uncommon capabilities that can be integrated into 
the scan workflows.  Each section in this part of the manual describes any source control 
system-specific capabilities integrated into the scanning workflow.  

This section will describe common elements for each SCM that are integrated into the
scanning workflow.

\section{Project Naming}

In the event a scan is executed and a matching project does not exist, a new project
is created prior to the scan.  By default, the project naming convention follows the same automatic
naming convention that is used when importing projects into \cxone using the 
Code Repository Integration.

If the default project name format is not acceptable, Chapter \ref{sec:project-naming} describes
an advanced configuration that allows for project names to be dynamically created.

\section{Project Tags}

In addition to any configured default project tags, Table \ref{tab:project-tags} shows
additional tags that are assigned to a project upon creation.

\begin{table}[ht]
    \caption{Project Tags}  
    \label{tab:project-tags}      
    \begin{tabularx}{\textwidth}{ll}
        \toprule
        \textbf{Tag Name} & \textbf{Value} \\
        \midrule
        \texttt{service} & \makecell[l]{The configured service name, as described in Section \ref{sec:yaml-config}, that 
        \\handled the scan workflow based on the matching route.}\\
        \midrule
        \texttt{cxone-flow} & \makecell[l]{The version of \cxoneflow that handled the scan orchestration.}\\
        \bottomrule
    \end{tabularx}
\end{table}


\section{Push Scan Tags}

When scan workflows are initiated by a "push" (e.g. a code commit to a protected branch),
the scans are assigned the tags as described in Table \ref{tab:push-scan-tags}.  These
tags are added in addition to any configured default scan tags.

\begin{table}[ht]
    \caption{Push Scan Tags}  
    \label{tab:push-scan-tags}      
    \begin{tabularx}{\textwidth}{ll}
        \toprule
        \textbf{Tag Name} & \textbf{Value} \\
        \midrule
        \texttt{commit} & \makecell[l]{The commit hash for the push the invoked the scan.}\\
        \midrule
        \texttt{workflow} & \makecell[l]{Always set to \textbf{push}.}\\
        \midrule
        \texttt{service} & \makecell[l]{The configured service name, as described in Section \ref{sec:yaml-config}, that 
        \\handled the scan workflow based on the matching route.}\\
        \midrule
        \texttt{cxone-flow} & \makecell[l]{The version of \cxoneflow that handled the scan orchestration.}\\
        \bottomrule
    \end{tabularx}
\end{table}


\section{Pull-Request Scan Tags}

Pull-Request scan tags have several tags with names matching those assigned by Push scan tags. 
Pull-Requests may follow a workflow enforced by the source control system; the state of the
Pull-Request workflow is captured in tags to the extent possible. It is not always possible
to capture the Pull-Request state accurately given it may change during a period of time
when it is not possible to update a scan's tags.  Table \ref{tab:pr-scan-tags} lists the
tags that can be observed for scans invoked by a Pull-Request event.


\begin{table}[ht]
    \caption{Pull-Request Scan Tags}  
    \label{tab:pr-scan-tags}      
    \begin{tabularx}{\textwidth}{ll}
        \toprule
        \textbf{Tag Name} & \textbf{Value} \\
        \midrule
        \texttt{commit} & \makecell[l]{The commit hash for the push the invoked the scan.}\\
        \midrule
        \texttt{workflow} & \makecell[l]{Always set to \textbf{pull-request}.}\\
        \midrule
        \texttt{service} & \makecell[l]{The configured service name, as described in Section \ref{sec:yaml-config}, that 
        \\handled the scan workflow based on the matching route.}\\
        \midrule
        \texttt{cxone-flow} & \makecell[l]{The version of \cxoneflow that handled the scan orchestration.}\\
        \midrule
        \texttt{pr-id} & \makecell[l]{The pull-request identifier that invoked the scan.}\\
        \midrule
        \texttt{pr-target} & \makecell[l]{The branch targeted by the pull-request.}\\
        \midrule
        \texttt{pr-status} & \makecell[l]{The review status of the pull-request.}\\
        \midrule
        \texttt{pr-state} & \makecell[l]{The state of the pull-request.}\\
        \bottomrule
    \end{tabularx}
\end{table}
