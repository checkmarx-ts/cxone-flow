GitHub uses one or more \textbf{Organization} logical units to separate zero or more
repositories into logical groupings. Each GitHub user is also a limited variation of an
\textbf{Organization} in that a user account is also a logical unit that can have
associated repositories.  Webhook configurations can be deployed at this scope.
When webhooks are deployed at the \textbf{Organization} scope, events
will be emitted for all repositories within the \textbf{Organization} logical unit.

Configuration of webhooks at the \textbf{Organization} scope is generally the preferred
method of webhook deployment for GitHub.

A \textbf{GitHub App} can be created that functions as a webhook deployment template.
The \textbf{GitHub App} can be deployed at the \textbf{Organization} scope using the
GitHub user interface.  This is the preferred method of defining and executing the webhook deployment.

Webhooks can be deployed at the scope of each \textbf{Repository} if desired.  The number of repositories
in a large enterprise generally makes deployment at the \textbf{Repository} scope useful
only for testing purposes.  
