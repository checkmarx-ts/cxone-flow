\section{2.1.1}

\begin{itemize}
    \item Added a kickoff API for use with \extlink{https://github.com/checkmarx-ts/cxone-flow-audit}{cxone-flow-audit}
          to iterate SCM repositories and perform an initial scan.
\end{itemize}


\section{2.1.0}

\begin{itemize}
    \item Fixed a branch naming bug that affected names of scanned branches that had "/" separated segments.
    \item Added Gitlab SCM support.
    \item Improved logging for feedback workflows.
    \item Fixed an issue with the resolver agent failing when no resolver options are provided.
    \item A breaking configuration change was implemented such that \texttt{repo-match} values that are a wildcard
          match for any value are no longer allowed.  This is to prevent potential SSRF attacks exposing SCM credentials
          if the SCM shared secret is compromised.  The route match should, at minimum, specify common parts of the FQDN
          portion of the Git clone URL. Please consult the documentation for more information.
\end{itemize}


\section{2.0.0}

This version is compatible with any configuration YAML that works with version 1.3.  If you are using an external message queue (e.g. not the
message queue running internally in the CxOneFlow container), please see the documentation regarding "Upgrading CxOneFlow from 1.x to 2.x".

\begin{itemize}
    \item Refactored to use the shared library \extlink{https://github.com/checkmarx-ts/cxone-async-api}{\textbf{cxone-async-api}}\space instead
    of an embedded copy of the \texttt{cxone\_api} code.
    \item Added support for dependency resolution execution using SCA Resolver prior to executing a scan.
    \item The \texttt{cxoneflow-resolver-agent} is available for Debian platforms.  This is the agent that executes SCA Resolver.
    \item The \texttt{cxoneflow-resolver-agent} supports SCA Resolver execution via containers using the \toolkit.
    \item The server YAML configuration documentation has been updated to add clarity about the YAML configuration format.
\end{itemize}


\section{1.3}

\begin{itemize}
    \item Now supports GitHub Enterprise and GitHub Cloud
    \item Documentation updates
\end{itemize}

\section{1.2}

\begin{itemize}
    \item Fixed problem where pooled connections to the SCM would cause PR failures when the network routing 
    fabric discarded the connection route.
    \item BREAKING CHANGE - Added a new configuration option for the CxOneFlow server base URL used to host image
    artifacts used in PR feedback comments.
    \item Fixed an issue with \texttt{ssl-verify} not working with self-signed CAs.
    \item Documentation updates.
\end{itemize}


\section{1.1}

\begin{itemize}
    \item Added more verbose logging to aid in troubleshooting feedback workflows.
    \item Fixed an issue where the feedback scan polling parameters were ignored in some configurations.
    \item Added result summary to PR feedback comments.
    \item If detailed PR feedback comment content exceeds the SCM's comment limit, the comment will be limited to the summary content.
    \item Fixed an issue with the scan permalink published in the PR feedback comments.
\end{itemize}


\section{1.0}

\begin{itemize}
    \item Stabilize Azure DevOps Enterprise Cloud basic authorization implementation.
    \item ADO service hook test ping now fails if the correct shared secret is not provided.
\end{itemize}


\section{0.2 - Beta Release}
   

\begin{itemize}
    \item Support for Azure DevOps Enterprise Cloud.
    \item Pull-request feedback.
\end{itemize}

\section{0.1 - Initial Beta Release}
   

\begin{itemize}
    \item Support for BitBucket Data Center 8.19.2\footnote{Testing was performed against 8.19.2 but it is possible that earlier versions will work.}
    \item Support for Azure DevOps Enterprise 2022.1\footnote{Testing was performed against 2022.1 but it is possible that earlier versions will work.}
    \item Webhook Endpoint support for BitBucket Data Center (BBDC) and Azure DevOps Enterprise (ADOE).
    \item Scan on push to protected branches.
    \item Scan on pull-request targetting protected branches.
    \item Pull-request workflow scan tagging.
\end{itemize}