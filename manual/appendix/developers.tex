\chapter{\cxoneflow Developer Information}

The development is performed on Ubuntu using Visual Studio Code.  This is not 
strictly required but any instructions for quick starting the development environment 
will be for using Ubuntu and VSCode. If you have different development tooling you'd like
to use, you'll need to adapt these instructions to fit your tooling.

\noindent\\The target version of Python is currently 3.12.  Using versions prior to 3.12 will likely result in
runtime errors.

\section{Secrets}

\textbf{DO NOT COMMIT SECRETS}

\noindent\\In general, it is a bad idea to commit secrets to the repository.  The \texttt{.gitignore} file is configured
to ignore folders named \texttt{secrets} and YAML files in the root of the code directory.  This 
can be circumvented or other files containing secrets can be inadvertently committed.

\noindent\\If you've committed secrets but not pushed to the public GitHub repository, you can edit the commit history
or merge your clean code into a new clone.  If you've pushed a secret into the public repository, please notify
the other maintainers so the impact can be assessed.

\section{Running with a Debugger}

The VSCode debug menu has a \texttt{Flask} debug task that will run a single endpoint locally.  By default,
the code will attempt to open \texttt{./config.yaml} for the YAML configuration.  Provide a
\texttt{config.yaml} file in the root of your development environment or set the appropriate environment
variables with the path to your \texttt{config.yaml} file.


\section{Installing \LaTeX\space for Documentation}

The documentation is written in LaTeX, so enabling VSCode to lint, compile, and preview
LaTeX requires some configuration.

\subsection{\LaTeX\space Setup}

Follow these steps to install \LaTeX.

\begin{enumerate}
    \item In VSCode, install the "LaTeX Workshop" plugin by James Yu

    \item Install TexLive direct from the website.  This is required since
    most Debian package repositories have an older version of TexLive.
    \begin{enumerate}
        \item Download instructions: \href{https://www.tug.org/texlive/acquire-netinstall.html}{https://www.tug.org/texlive/acquire-netinstall.html}
        \item Install instructions: \href{https://www.tug.org/texlive/quickinstall.html}{https://www.tug.org/texlive/quickinstall.html}
    \end{enumerate}

    \item At the end of the install, it will instruct you to update `PATH`, `MANPATH`, and `INFOPATH`.
    Set these in `~/.bashrc`, close your shell and re-open it to get the new environment variables.

    \item Execute \texttt{sudo \$(which texconfig) rehash}

\end{enumerate}

\noindent\\If \LaTeX\space is installed correctly, opening any of the \texttt{.tex} files will prove the ability to
compile and preview the manual PDF.
