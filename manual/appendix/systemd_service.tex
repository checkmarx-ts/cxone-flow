\chapter{Running \cxoneflow as a \texttt{systemd} Service}


Running \cxoneflow as a \texttt{systemd} service on Linux is one method that
will allow the container to be started automatically when the hosting system
is rebooted.  The \texttt{systemd} service can then be used to start and stop the
\cxoneflow services.  This section details an example configuration that can be adapted
to your environment.


\section{Obtain and Store Files}

In the directory \texttt{/opt/cxoneflow}, collect the configuration YAML, PEM encoded SSL certificate, PEM encoded SSL certificate private key,
and any secret\footnote{If planning to map secrets to the container using other methods, 
writing the files is not required.} files referenced in the configuration YAML.  The following
ownership and permissions files are suggested:

\begin{itemize}
    \item Set the ownership to \texttt{root:docker} with the command \texttt{chown -R root:docker /opt/cxoneflow}
    \item Set the permissions of \texttt{/opt/cxoneflow} using the command \texttt{chmod 750 /opt/cxoneflow}
    \item Set the permissions of the files in \texttt{/opt/cxoneflow} using the command
    \texttt{chmod -R 740 /opt/cxoneflow/*}
\end{itemize}


\section{Container Creation}

Creating a reusable container with the name \textbf{cxoneflow} can be done with the following
command:

\begin{code}{Container Creation Command}{}{}
docker create --rm --name "cxoneflow" -p 443:8443 --env-file runtime.env \
  -v /path/to/cert.pem:/opt/cxone/certs/mycert.pem \
  -v /path/to/private.key.pem:/opt/cxone/certs/my.private.key.pem \
  -v /path/to/config.yaml:/opt/cxone/config.yaml \
  -v /path/to/secrets:/run/secrets \
  ghcr.io/checkmarx-ts/cxone/cxone-flow:latest
\end{code}

\noindent\\The command creates a container with the following properties:

\begin{itemize}
    \item The container with the name \textbf{cxoneflow} is created.
    \item The \texttt{----rm} automatically removes the container named \textbf{cxoneflow} if
    it exists.
    \item Port 443 is mapped to port 8443 of the container to support secure communication with
    the \cxoneflow endpoint.
    \item Runtime environment variables (as described in Section \ref{sec:runtime-config}) 
    located in the file \texttt{runtime.env} are injected into the container's environment.
    \item Maps the following files from the local \texttt{/path/to/} path to files for the SSL
    certificate, configuration YAML, and secrets
    \footnote{This is assuming the secrets are not already mapped to the container using other methods.}.
\end{itemize}

\noindent\\It is suggested that the command is incorporated into a shell script so it may be
invoked to recreate the container during a version upgrade.

\section{Service Definition}

A \texttt{systemd} service definition should be created using a file named, for example,
\\\texttt{/opt/cxoneflow/cxoneflow.service}.  The service definition should have contents
similar to the following:

\begin{code}{/opt/cxoneflow/cxoneflow.service}{}{}
[Unit]
Description=CxOneFlow container
Requires=docker.service
After=docker.service

[Service]
Restart=always
ExecStart=/usr/bin/docker start -a cxoneflow
ExecStop=/usr/bin/docker stop cxoneflow
Type=exec

[Install]
WantedBy=default.target
        
\end{code}

\noindent\\The service definition allows \texttt{systemd} to manage the start
and stop of the \cxoneflow service in response to system startup and shutdown.


\section{Enable and Start the \cxoneflow Service}

The following commands will enable the service and start \cxoneflow:

\begin{code}{}{}{}
sudo systemctl enable /opt/cxoneflow/cxoneflow.service
sudo systemctl start cxoneflow
\end{code}

