\chapter{\cxoneflowtext\space Security Considerations}\label{sec:cxoneflow-security}


\section{SCM Shared Secret}

The SCM event configurations all utilize a "shared secret" token that is used to validate that
a webhook event is emitted from a trusted source SCM.  In some SCMs, the token is used to
demonstrate mutual knowledge if the value using HMAC signing of the payload.  In most
of the SCMs, however, the value is sent in either plaintext or base-64 encoded text.  Anyone
monitoring the communication between the SCM and \cxoneflow can obtain the shared secret
value when it is sent as part of the event payload.

The shared secret is often the only mechanism available to prevent someone from
spoofing the payload of an event.  The unfortunate truth is that setting up
SCM webhooks will require this secret to be revealed to several people. The more people
that know the secret, the more likely it is that it will be exposed.

The \cxoneflow endpoint is designed to receive JSON payloads as a machine-to-machine 
communication but anyone with a network connection path to \cxoneflow can send any
payload.  The event payload formats and secret verification methods are well
known.  If a threat-actor had the shared secret, this would not be desirable.

The \cxoneflow configuration uses a regular expression to match the service definition
that is intended to handle a received webhook event.  The \texttt{repo-match} element
configured with an all-matching regular expression such as \texttt{.*} means that the
service definition will handle an event for any repository clone URL presented in the
event.  If a threat actor with the shared secret were to send a modified payload
containing a clone URL of their choosing, it would be accepted for handling by
the matching service definition.  This would potentially allow exposure of the
clone credentials via a SSRF attack.

The following recommendations can be used to mitigate and possibly prevent any
issues that may occur due to exposure of the shared secret.

\subsection{Mitigating Shared Secret Exposure}


\subsubsection{Avoid All-Matching Regular Expressions}

To prevent an SSRF scenario due to an all-matching regular expression, 
\cxoneflow as of version 2.1.0 will not start if a configured \texttt{repo-match} element
will match arbitrary values. Using a regular expression that matches the transport and FQDN
of clone URLs expected in each event payload will prevent events from being matched to a 
service definition.  As an example, most regular expressions would be in a format similar to the one below:

\begin{code}{General Clone URL Regular Expression}{}{}
^http(s)?:(\/){2}my\.scm\.corp\.com.*
\end{code}

In the case of cloud-hosted Azure DevOps, an slightly different regular expression
is needed.  The reason for this is that pull request events emitted from the multi-tenant ADO cloud have
the organization name in the clone URL.  Below is an example of an Azure DevOps cloud
service endpoint:

\begin{code}{Minimal YAML Configuration}{Azure DevOps Cloud}{}
  secret-root-path: /run/secrets
  server-base-url: https://cxoneflow.mydomain.com:8443/
  
  adoe:
      - service-name: MyADO
        repo-match: ^http(s)?:(\/){2}(.+@)?dev\.azure\.com.*
        connection:
        base-url: https://scm.corp.com/
        shared-secret: scm-shared-secret
        api-auth:
          username: scm-username-secret
          password: scm-password-secret
        cxone:
          tenant: mytenant
          oauth:
            client-id: my-cxone-client-id
            client-secret: my-cxone-client-secret
          iam-endpoint: US
          api-endpoint: US
\end{code}
  
\subsubsection{Limit Outbound Network Connectivity of the \cxoneflowtext\space Endpoint}

When \cxoneflow is operating, there are only very few connections to external machines
needed.  These are generally:

\begin{itemize}
  \item The IAM and API URLs hosting the \cxone tenant.
  \item 
\end{itemize}