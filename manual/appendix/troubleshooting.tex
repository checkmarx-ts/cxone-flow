\chapter{Troubleshooting}

If \cxoneflow does not appear to be operating as expected, the following
troubleshooting steps may help to understand the reason:

\begin{itemize}
    \item The environment variable \texttt{LOG\_LEVEL} can be set to \texttt{DEBUG}
    to emit trace information as \cxoneflow is running.

    \item Navigate to the \texttt{ping} endpoint in your browser to see the 
    \texttt{pong} response using: \texttt{http://<server>/ping}
    
    \item The \cxoneflow logs are streamed to the container's stdout.  It can be
    viewed by running the container interactively or using a command such as
    \texttt{docker logs -f <container name>}

    \item Nginx and Gunicorn logs can be found in \texttt{/var/log} on the container
    image.  A shell on the container can be opened using a command such as
    \texttt{docker exec -it <container name> bash}

    \item For Git cloning issues, the setting environment variables \texttt{GIT\_TRACE=1} and
    \texttt{GIT\_CURL\_VERBOSE=1} may log useful troubleshooting information.

\end{itemize}

\noindent\\Exception stack traces emitted in the \cxoneflow logs can generally
indicate any issues it is encountering.  Many issues are likely to be configuration
related.