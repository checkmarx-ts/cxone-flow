\chapter{Upgrading}

\section{Switching to an external RabbitMQ Instance}\label{sec:rmq-switch}

The \cxoneflow container has an internal RabbitMQ instance that is used to orchestrate messaging 
for \hyperref[part:workflows]{asynchronous workflows} supported by \cxoneflow.  If the configuration
does not specify \hyperref[sec:yaml-config]{an external AMQP endpoint}, the internal RabbitMQ instance
is used.  When the running \cxoneflow container is deleted, the messages contained in the internal
RabbitMQ instance are also deleted.

When upgrading to a new version of \cxoneflow or changing the AMQP connection configuration, it is advised
to let the existing \cxoneflow container continue to run while the agents process the pending messages.\footnote{To view 
the queue message counts, execute the command \texttt{rabbitmqctl list\_queues} in a shell on the \cxoneflow container.}  
When the queues no longer have any pending or processing messages, the old \cxoneflow instance can be terminated.

\section{From 1.x to 2.x}

The configuration YAML for version 1.3 is compatible with version 2.x.  If the \cxoneflow instance uses only the internal
RabbitMQ instance, then the need to drain the messages in the internal queues as described in Section \ref{sec:rmq-switch}
applies.  If not using an external AMQP endpoint with \cxoneflow prior to version 2.x, nothing further is required for 
upgrade activities.

Version 2.x introduced a change to the naming of exchanges and queues.  The version 2.x naming changes will not clash with
names of exchanges and queues for prior versions in the same host, but version 2.x workflow agents will not process messages
from the 1.x named queues.  In this case, it is advised to use the following procedure when upgrading:

\begin{enumerate}
  \item Create a copy of the \cxoneflow configuration for 1.x and modify it to be compatible with version 2.x.
  \item Create one or more instances of \cxoneflow 2.x that run in parallel with \cxoneflow 1.x instances.
  \item Leave the \cxoneflow 1.x instances running so that the messages are drained from the version 1.x queues.
  \item Make a DNS change that points to the version 2.x \cxoneflow instance so that all new events are handled by
  the version 2.x endpoints.
  \item When the version 1.x queues are drained, the \cxoneflow version 1.x instances can be terminated.
\end{enumerate}



\section{From 1.x to 1.3}

Version 1.3 introduced the required YAML root element \texttt{server-base-url}.  This element needs to be added to the configuration YAML
or the \cxoneflow server will not start.  Please refer to Section \ref{sec:yaml-root} for more information about
the \texttt{server-base-url} root element.
